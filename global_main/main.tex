\documentclass[12pt, a4paper]{article}
\usepackage[T2A]{fontenc}
\usepackage[utf8]{inputenc}
\usepackage[russian]{babel}
\usepackage{amsmath, amssymb}
\usepackage{graphicx}
\usepackage{geometry}
\geometry{left=20mm, right=15mm, top=20mm, bottom=20mm}

\title{Автоколебания. Стоячие волны.}
\author{Малышев Павел}
\date{}

\begin{document}

\maketitle

\section{Содержание}
\begin{enumerate}
    \item Вступление.
    \item Затухания и вынужденные колебания.
    \item Автоколебания.
    \item Стоячие волны.
    \item Демонстрационный опыт.
    \item Приложение, список литературы.
\end{enumerate}
\section{Введение}

Колебательные процессы встречаются практически во всех разделах физики — от механики и акустики до радиотехники и оптики.
В простейших моделях мы изучаем либо свободные колебания, которые со временем затухают из-за потерь, либо вынужденные колебания,
где периодическое внешнее воздействие задаёт частоту системы.
\\
\\
Однако в реальных физических системах существует третий, принципиально иной класс колебаний — \textbf{автоколебания}.
Автоколебания возникают при постоянном подводе энергии и поддерживаются самой системой за счёт положительной обратной связи.
В отличие от вынужденных колебаний, здесь частота и амплитуда не навязываются извне,
а определяются внутренними свойствами системы и её нелинейностями.
Такие колебания устойчивы и не зависят от начальных условий: система сама выходит на установившийся режим, который соответствует балансу между подводом и потерями энергии.
\\
\\
Во многих физических ситуациях автоколебания возникают в протяжённых системах,
обладающих собственными резонансными модами.
В этом случае автоколебательный механизм приводит к возбуждению стоячих волн, форма и частота которых определяются граничными условиями.
Типичными примерами являются струны, трубы, резонаторы, а также колебания упругих тел.
\\
\\
В своём сообщении я рассмотрю теоретические основы автоколебаний и стоячих волн и покажу,
как оба этих явления проявляются в простом и наглядном примере — \textbf{скрипе мела о доску}.
\\
\\
Этот эффект представляет собой автоколебательный процесс,
возникающий из-за нелинейного трения, при котором возбуждается собственная стоячая волна в мелке, что и приводит к появлению характерного звука.


\section{Затухающие и вынужденные колебания}

Для понимания природы автоколебаний сначала рассмотрим поведение линейного гармонического осциллятора с потерями и внешним воздействием.

\subsection{Затухающие колебания}

Рассмотрим одномерный осциллятор массы $m$ с линейной силой упругости и вязким трением:
\[
m\ddot x + b\dot x + kx = 0.
\]
Введём стандартные обозначения
\[
\omega_0 = \sqrt{\frac{k}{m}}, \qquad \gamma = \frac{b}{2m}.
\]
Тогда уравнение движения принимает вид
\[
\ddot x + 2\gamma \dot x + \omega_0^2 x = 0.
\]

В случае слабого затухания ($\gamma < \omega_0$) решение имеет вид
\[
x(t) = A e^{-\gamma t}\cos(\omega t + \varphi),
\qquad
\omega = \sqrt{\omega_0^2 - \gamma^2}.
\]

Амплитуда колебаний экспоненциально убывает со временем:
\[
A(t) = A_0 e^{-\gamma t}.
\]
Это означает, что механическая энергия системы
\[
E = \frac{m\dot x^2}{2} + \frac{kx^2}{2}
\]
также убывает со временем, так как часть энергии рассеивается силой трения.

Таким образом, в линейной системе с потерями свободные колебания всегда затухают и не могут поддерживаться бесконечно долго.

\subsection{Вынужденные колебания}

Теперь добавим внешнюю периодическую силу:
\[
m\ddot x + b\dot x + kx = F_0 \cos(\Omega t).
\]
В стандартной форме:
\[
\ddot x + 2\gamma \dot x + \omega_0^2 x = \frac{F_0}{m}\cos(\Omega t).
\]

Решение этого уравнения состоит из суммы затухающей переходной части и установившегося решения. Нас интересует установившийся режим, который имеет вид
\[
x(t) = A(\Omega)\cos(\Omega t - \delta).
\]

Амплитуда вынужденных колебаний равна
\[
A(\Omega) =
\frac{F_0/m}
{\sqrt{(\omega_0^2 - \Omega^2)^2 + (2\gamma \Omega)^2}}.
\]

Фазовый сдвиг между силой и откликом системы определяется соотношением
\[
\tan \delta =
\frac{2\gamma \Omega}{\omega_0^2 - \Omega^2}.
\]

Максимум амплитуды достигается при частоте
\[
\Omega \approx \sqrt{\omega_0^2 - 2\gamma^2},
\]
что соответствует явлению резонанса.

\begin{figure}[h]
    \centering
    \includegraphics[width=0.7\linewidth]{images/grph_zatuh_koleb.png}
    \caption{Затухающие колебания линейного осциллятора:
    амплитуда экспоненциально убывает из-за потерь энергии.}
    \end{figure}

Важно подчеркнуть, что в режиме вынужденных колебаний:
\begin{itemize}
  \item частота колебаний \textbf{совпадает} с частотой внешней силы $\Omega$;
  \item энергия, теряемая на трение, в точности компенсируется работой внешней силы за период;
  \item при исчезновении внешнего воздействия колебания снова затухают.
\end{itemize}

\subsection{Ограниченность линейной модели}

Затухающие и вынужденные колебания показывают, что:
\begin{itemize}
  \item без подвода энергии колебания затухают;
  \item при периодическом подводе энергии система колеблется только на навязанной частоте;
  \item амплитуда определяется параметрами внешней силы.
\end{itemize}

Следовательно, линейная теория не описывает ситуацию, когда система:
\begin{itemize}
  \item получает энергию от \textbf{постоянного} источника;
  \item сама выбирает частоту колебаний;
  \item выходит на устойчивый режим \textbf{независимо} от начальных условий.
\end{itemize}

Именно такие режимы реализуются в автоколебательных системах, к рассмотрению которых мы переходим далее.

\section{Автоколебания}

% ТУТ АВТОКОЛЕБАНИЯ А ПОТОМ СТОЯЧИЕ ВОЛНЫ



\section{Скрип мела о доску как автоколебательный процесс и возбуждение стоячей волны}

Рассмотрим скрип мела о доску как наглядный пример, в котором одновременно реализуются
механизм автоколебаний и возбуждение собственной стоячей волны в упругом теле.

\subsection{Механизм скрипа и автоколебания}

При движении мела по доске сила трения не является постоянной и линейной функцией скорости.
Важную роль играет так называемый режим \emph{stick--slip} (прилипание--скольжение).

Физическая картина процесса следующая:
\begin{itemize}
  \item при движении мела он периодически \emph{прилипает} к поверхности доски;
  \item в фазе прилипания в мелке накапливается упругая деформация;
  \item при достижении предельной силы трения происходит \emph{срыв} и быстрое скольжение;
  \item после срыва сила трения уменьшается, мел возвращается назад, и процесс повторяется.
\end{itemize}

Таким образом, при постоянной скорости движения руки реализуется периодический процесс,
при котором энергия трения частично преобразуется в энергию механических колебаний.
Это соответствует автоколебательному механизму: источник энергии постоянен,
а периодичность возникает самопроизвольно за счёт нелинейной обратной связи.

Упрощённо движение мела можно описать уравнением вида
\[
m\ddot x + c\dot x + kx = F_{\mathrm{тр}}(\dot x),
\]
где сила трения $F_{\mathrm{тр}}(\dot x)$ является нелинейной функцией скорости.
В определённом диапазоне скоростей такая зависимость приводит к эффекту
``отрицательного трения'', что вызывает самовозбуждение колебаний.

\subsection{Собственная мода мела и стоячая волна}

Мел является упругим стержнем конечной длины и, следовательно,
обладает собственными колебательными модами.
При удержании мела рукой один его конец оказывается практически зафиксирован,
в то время как другой конец остаётся относительно свободным.

В первом приближении такая система эквивалентна резонатору с граничными условиями:
\begin{itemize}
  \item у удерживаемого конца --- узел смещения;
  \item у свободного конца --- пучность смещения.
\end{itemize}

Для такой конфигурации основная мода соответствует четвертьволновому резонансу:
\[
L \approx \frac{\lambda}{4},
\]
где $L$ --- длина мела, $\lambda$ --- длина упругой волны в материале мела.

Соответствующая собственная частота порядка
\[
f \approx \frac{u}{4L},
\]
где $u$ --- скорость распространения упругих волн в мелке.

Именно эта собственная частота выбирается автоколебательным механизмом:
трение подкачивает энергию наиболее эффективно на резонансной моде,
что приводит к возбуждению стоячей волны вдоль мела.

\subsection{Физический итог}

Таким образом, скрип мела о доску представляет собой автоколебательный процесс,
в котором:
\begin{itemize}
  \item нелинейное трение реализует механизм самовозбуждения (stick--slip);
  \item мел как упругий стержень играет роль резонатора;
  \item в мелке возбуждается собственная стоячая волна;
  \item излучение этой волны в воздух воспринимается как резкий звук.
\end{itemize}

Отсутствие скрипа у короткого мела объясняется тем,
что его собственные частоты выходят из слышимого диапазона,
и автоколебательный режим становится неэффективным.


\end{document}
