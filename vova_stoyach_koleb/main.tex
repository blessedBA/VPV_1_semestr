\documentclass[12pt, a4paper]{article}
\usepackage[T2A]{fontenc}
\usepackage[utf8]{inputenc}
\usepackage[russian]{babel}
\usepackage{amsmath, amssymb}
\usepackage{graphicx}
\usepackage{geometry}
\usepackage{hyperref}
\usepackage{pgfplots}
\usepackage{booktabs}
\usepackage{caption}
\usepackage{subcaption}
\usepackage{array}
\usepackage{tikz}
\usepackage{float}
\pgfplotsset{compat=1.18}

\geometry{left=20mm, right=15mm, top=20mm, bottom=20mm}

\title{Автоколебания. Стоячие волны. Модель скрипа мела.}
\author{Малышев Павел}
\date{}

\begin{document}

\maketitle

\section*{Содержание}
\begin{enumerate}
    \item Введение.
    \item Затухающие и вынужденные колебания.
    \item Автоколебания.
    \item Стоячие волны.
    \item Экспериментальная проверка модели.
    \item Влияние состояния доски на скрипучесть мела.
    \item Преобразование механических колебаний в звук.
    \item Заключение.
    \item Источники информации.
\end{enumerate}

\section{Введение}

Колебательные процессы встречаются практически во всех разделах физики — от механики и акустики до радиотехники и оптики.
В простейших моделях мы изучаем либо свободные колебания, которые со временем затухают из-за потерь, либо вынужденные колебания,
где периодическое внешнее воздействие задаёт частоту системы.
\\
\\
Однако в реальных физических системах существует третий, принципиально иной класс колебаний — \textbf{автоколебания}.
Автоколебания возникают при постоянном подводе энергии и поддерживаются самой системой за счёт положительной обратной связи.
В отличие от вынужденных колебаний, здесь частота и амплитуда не навязываются извне,
а определяются внутренними свойствами системы и её нелинейностями.
Такие колебания устойчивы и не зависят от начальных условий: система сама выходит на установившийся режим, который соответствует балансу между подводом и потерями энергии.
\\
\\
Во многих физических ситуациях автоколебания возникают в протяжённых системах,
обладающих собственными резонансными модами.
В этом случае автоколебательный механизм приводит к возбуждению стоячих волн, форма и частота которых определяются граничными условиями.
Типичными примерами являются струны, трубы, резонаторы, а также колебания упругих тел.
\\
\\
В своём сообщении я рассмотрю теоретические основы автоколебаний и стоячих волн и покажу,
как оба этих явления проявляются в простом и наглядном примере — \textbf{скрипе мела о доску}.
\\
\\
Этот эффект представляет собой автоколебательный процесс,
возникающий из-за нелинейного трения, при котором возбуждается собственная стоячая волна в мелке, что и приводит к появлению характерного звука.

\section{Затухающие и вынужденные колебания}

Для понимания природы автоколебаний сначала рассмотрим поведение линейного гармонического осциллятора с потерями и внешним воздействием.

\subsection{Затухающие колебания}

Рассмотрим одномерный осциллятор массы $m$ с линейной силой упругости и вязким трением:
\[
m\ddot x + b\dot x + kx = 0.
\]
Введём стандартные обозначения
\[
\omega_0 = \sqrt{\frac{k}{m}}, \qquad \gamma = \frac{b}{2m}.
\]
Тогда уравнение движения принимает вид
\[
\ddot x + 2\gamma \dot x + \omega_0^2 x = 0.
\]

В случае слабого затухания ($\gamma < \omega_0$) решение имеет вид
\[
x(t) = A e^{-\gamma t}\cos(\omega t + \varphi),
\qquad
\omega = \sqrt{\omega_0^2 - \gamma^2}.
\]

Амплитуда колебаний экспоненциально убывает со временем:
\[
A(t) = A_0 e^{-\gamma t}.
\]
Это означает, что механическая энергия системы
\[
E = \frac{m\dot x^2}{2} + \frac{kx^2}{2}
\]
также убывает со временем, так как часть энергии рассеивается силой трения.

Таким образом, в линейной системе с потерями свободные колебания всегда затухают и не могут поддерживаться бесконечно долго.

\subsection{Вынужденные колебания}

Теперь добавим внешнюю периодическую силу:
\[
m\ddot x + b\dot x + kx = F_0 \cos(\Omega t).
\]
В стандартной форме:
\[
\ddot x + 2\gamma \dot x + \omega_0^2 x = \frac{F_0}{m}\cos(\Omega t).
\]

Решение этого уравнения состоит из суммы затухающей переходной части и установившегося решения. Нас интересует установившийся режим, который имеет вид
\[
x(t) = A(\Omega)\cos(\Omega t - \delta).
\]

Амплитуда вынужденных колебаний равна
\[
A(\Omega) =
\frac{F_0/m}
{\sqrt{(\omega_0^2 - \Omega^2)^2 + (2\gamma \Omega)^2}}.
\]

Фазовый сдвиг между силой и откликом системы определяется соотношением
\[
\tan \delta =
\frac{2\gamma \Omega}{\omega_0^2 - \Omega^2}.
\]

Максимум амплитуды достигается при частоте
\[
\Omega \approx \sqrt{\omega_0^2 - 2\gamma^2},
\]
что соответствует явлению резонанса.

\begin{figure}[H]
    \centering
    \begin{tikzpicture}
        \begin{axis}[
            width=0.7\linewidth,
            height=6cm,
            xlabel={$t$, с},
            ylabel={$x(t)$},
            domain=0:10,
            samples=200,
            axis lines=middle,
            xmin=0, xmax=10,
            ymin=-1.2, ymax=1.2,
            grid=both,
            legend pos=north east
        ]
            \addplot[blue, thick] {exp(-0.2*x)*cos(deg(3*x))};
            \addlegendentry{Затухающие колебания};
            \addplot[red, dashed, thick] {exp(-0.2*x)};
            \addplot[red, dashed, thick] {-exp(-0.2*x)};
            \addlegendentry{Огибающая};
        \end{axis}
    \end{tikzpicture}
    \caption{Затухающие колебания линейного осциллятора: амплитуда экспоненциально убывает из-за потерь энергии.}
\end{figure}

Важно подчеркнуть, что в режиме вынужденных колебаний:
\begin{itemize}
  \item частота колебаний \textbf{совпадает} с частотой внешней силы $\Omega$;
  \item энергия, теряемая на трение, в точности компенсируется работой внешней силы за период;
  \item при исчезновении внешнего воздействия колебания снова затухают.
\end{itemize}

\subsection{Ограниченность линейной модели}

Затухающие и вынужденные колебания показывают, что:
\begin{itemize}
  \item без подвода энергии колебания затухают;
  \item при периодическом подводе энергии система колеблется только на навязанной частоте;
  \item амплитуда определяется параметрами внешней силы.
\end{itemize}

Следовательно, линейная теория не описывает ситуацию, когда система:
\begin{itemize}
  \item получает энергию от \textbf{постоянного} источника;
  \item сама выбирает частоту колебаний;
  \item выходит на устойчивый режим \textbf{независимо} от начальных условий.
\end{itemize}

Именно такие режимы реализуются в автоколебательных системах, к рассмотрению которых мы переходим далее.

\section{Автоколебания}

\subsection{Определение}

\textbf{Автоколебания} — это незатухающие колебания, которые:
\begin{itemize}
    \item поддерживаются за счёт \textbf{постоянного (непериодического) источника энергии},
    \item возникают \textbf{без внешнего периодического воздействия},
    \item имеют частоту, \textbf{определяемую самой системой}.
\end{itemize}

Принципиально важно, что система сама преобразует постоянный приток энергии в периодическое движение.

\subsection{Сравнительная таблица типов колебаний}

\begin{center}
\begin{tabular}{|c|c|c|}
\hline
Тип колебаний & Источник энергии & Частота \\
\hline
Свободные & начальный толчок & собственная \\
Затухающие & начальный толчок & собственная \\
Вынужденные & периодическая сила & навязанная \\
\textbf{Автоколебания} & \textbf{постоянный источник} & \textbf{собственная} \\
\hline
\end{tabular}
\end{center}

\subsection{Общая математическая модель}

Стандартное уравнение автоколебательной системы имеет вид
\[
\ddot{x} + \omega_0^2 x = F(x,\dot{x}),
\]
где:
\begin{itemize}
    \item $x(t)$ — обобщённая координата,
    \item $\omega_0$ — собственная частота линейной системы,
    \item $F(x,\dot{x})$ — \textit{нелинейная} активная сила.
\end{itemize}

\textbf{Важно:} при линейной зависимости $F$ от $x$ и $\dot{x}$ автоколебания невозможны.

\subsection{Осциллятор Ван-дер-Поля}

Классический пример автоколебательной системы:
\[
\ddot{x} - \mu (1 - x^2)\dot{x} + \omega_0^2 x = 0,
\]
где $\mu > 0$ — параметр нелинейности.

\begin{itemize}
    \item При $|x| \ll 1$: $(1 - x^2) \approx 1$ — отрицательное трение.
    \item При $|x| \gg 1$: трение становится положительным.
\end{itemize}

В результате система выходит на устойчивый режим колебаний.

\begin{figure}[H]
    \centering
    \begin{tikzpicture}
        \begin{axis}[
            width=0.5\linewidth,
            height=6cm,
            xlabel={$x$},
            ylabel={$\dot{x}$},
            domain=-2.5:2.5,
            samples=200,
            axis lines=middle,
            xmin=-2.5, xmax=2.5,
            ymin=-2.5, ymax=2.5,
            grid=both
        ]
            \addplot[blue, thick, smooth] {sqrt(1-x^2)};
            \addplot[blue, thick, smooth] {-sqrt(1-x^2)};
            \node at (axis cs: 1.8,2.0) {Предельный цикл};
        \end{axis}
    \end{tikzpicture}
    \caption{Осциллятор Ван-Дер Поля: фазовый портрет с предельным циклом.}
\end{figure}

\subsection{Предельный цикл}

В фазовом пространстве $(x,\dot{x})$ автоколебания соответствуют
\textbf{устойчивому предельному циклу}.

\textbf{Свойства предельного цикла:}
\begin{itemize}
    \item траектории из окрестности притягиваются к нему,
    \item амплитуда колебаний не зависит от начальных условий,
    \item режим является динамически устойчивым.
\end{itemize}

\subsection{Энергетический баланс}

За один период автоколебаний выполняется условие
\[
\oint F \dot{x}\,dt = 0.
\]

Это означает, что:
\begin{itemize}
    \item на части периода энергия подкачивается,
    \item на части — рассеивается,
    \item в среднем за период баланс равен нулю.
\end{itemize}

Автоколебания — это не отсутствие потерь, а их компенсация.

\subsection{Физические примеры}

\begin{itemize}
    \item \textbf{Скрип мела}: трение типа \emph{stick--slip}, возбуждение собственной моды стержня.
    \item \textbf{Свист чайника}: поток газа + акустический резонатор.
    \item \textbf{Электронные генераторы}: отрицательное сопротивление активного элемента.
    \item \textbf{Биологические ритмы}: автогенераторы, аналогичные осциллятору Ван-дер-Поля.
    \item \textbf{Листья}: колебания листьев растений под действием равномерного потока воздуха.
\end{itemize}

\subsection{Ключевые выводы}

\begin{enumerate}
    \item Автоколебания невозможны без нелинейности.
    \item Источник энергии \textbf{не} является периодическим.
    \item Частота определяется параметрами системы.
    \item Существует устойчивый предельный цикл.
    \item Автоколебания принципиально отличаются от резонанса.
\end{enumerate}

\section{Стоячие волны}

Стоячая волна — фундаментальное физическое явление, возникающее при интерференции двух идентичных волн,
распространяющихся навстречу друг другу. В отличие от бегущей волны, стоячая волна не переносит энергию в пространстве,
а характеризуется устойчивым распределением амплитуды колебаний, состоящим из чередующихся максимумов (пучностей) и минимумов (узлов).

\subsection{Гармонические колебания}

Любой волновой процесс основан на колебаниях. Простейшим и фундаментальным типом являются \textbf{гармонические колебания}, описываемые уравнением:

\begin{equation}
    y(t) = A \cos(\omega t + \varphi_0)
    \label{eq:simple_oscillation}
\end{equation}

где:
\begin{itemize}
    \item $y(t)$ — мгновенное значение колеблющейся величины (смещение, давление),
    \item $A$ — \textbf{амплитуда} колебаний (максимальное отклонение),
    \item $\omega = 2\pi f$ — \textbf{циклическая (круговая) частота},
    \item $f$ — частота колебаний в Герцах (Гц),
    \item $\varphi_0$ — \textbf{начальная фаза}.
\end{itemize}

Период колебаний $T$, за который фаза изменяется на $2\pi$, связан с частотой: $T = 1/f = 2\pi/\omega$.

\subsection{Бегущая волна как перенос колебаний}

\textbf{Бегущая волна} - это волновое движение,
при котором поверхность равных фаз (фазовые волновые фронты) перемещается с конечной скоростью (постоянной для однородной среды).
Примерами могут служить упругие волны в стержне, столбе газа или жидкости, электромагнитная волна вдоль длинной линии.
\\
\\
Если гармоническое колебание, возникнув в одной точке упругой среды (воздух, твёрдое тело), передаётся соседним точкам, возникает \textbf{волна} — процесс распространения колебаний в пространстве с переносом энергии, но без переноса вещества.

Для \textbf{плоской гармонической бегущей волны}, распространяющейся вдоль положительного направления оси $x$, уравнение имеет вид:

\begin{equation}
    y(x, t) = A \cos(\omega t - kx + \varphi_0)
    \label{eq:traveling_wave_plus}
\end{equation}

Для волны, бегущей в противоположном направлении:

\begin{equation}
    y(x, t) = A \cos(\omega t + kx + \varphi_0)
    \label{eq:traveling_wave_minus}
\end{equation}

Здесь введён новый параметр:
\begin{itemize}
    \item $k = \dfrac{2\pi}{\lambda}$ — \textbf{волновое число}, где $\lambda$ — \textbf{длина волны} (минимальное расстояние между точками, колеблющимися в одинаковой фазе).
\end{itemize}

Аргумент косинуса $(\omega t \mp kx + \varphi_0)$ называется \textbf{фазой волны}. Постоянство фазы ($\omega t \mp kx = \text{const}$) определяет движение волнового фронта. Дифференцируя это условие по времени, получаем \textbf{фазовую скорость волны} $v$:

\[
\omega \, dt \mp k \, dx = 0 \quad \Rightarrow \quad \frac{dx}{dt} = \pm \frac{\omega}{k}
\]

\begin{equation}
    v = \frac{\omega}{k} = \lambda f
    \label{eq:wave_velocity}
\end{equation}

Скорость распространения волны зависит от свойств среды. Для \textbf{продольных упругих волн} в тонком стержне она определяется формулой:

\begin{equation}
    v = \sqrt{\frac{E}{\rho}}
    \label{eq:speed_in_rod}
\end{equation}
где:
\begin{itemize}
    \item $E$ — модуль Юнга материала,
    \item $\rho$ — плотность материала.
\end{itemize}

\subsection{Принцип суперпозиции}

\textbf{Принцип суперпозиции (наложения)} является фундаментальным для линейных волновых процессов: если в среде распространяется несколько волн, то результирующее смещение любой частицы среды в любой момент времени равно векторной сумме смещений, которые вызывала бы каждая из волн в отдельности.

Математически для двух волн $y_1(x,t)$ и $y_2(x,t)$:
\begin{equation}
    y_{\text{рез}}(x,t) = y_1(x,t) + y_2(x,t)
    \label{eq:superposition_principle}
\end{equation}

\subsection{Интерференция}

\textbf{Интерференция} — это частное следствие принципа суперпозиции, проявляющееся в образовании устойчивой картины усиления и ослабления колебаний в разных точках пространства при сложении \textbf{когерентных волн}. Волны называются когерентными, если они имеют \textbf{одинаковую частоту} и \textbf{постоянную во времени разность фаз}.

Результат интерференции зависит от разности фаз $\Delta \varphi$ складываемых волн в точке наблюдения:
\begin{itemize}
    \item \textbf{Усиление (максимум)}: $\Delta \varphi = 2\pi m$, где $m = 0, \pm 1, \pm 2, \dots$
    \item \textbf{Ослабление (минимум)}: $\Delta \varphi = \pi (2m + 1)$.
\end{itemize}

Для двух волн, приходящих в точку от разных источников, разность фаз определяется разностью расстояний до источников (\textbf{разностью хода} $\Delta L$): $\Delta \varphi = k \Delta L = \frac{2\pi}{\lambda} \Delta L$.

\subsection{Определение и общий вид}

\textbf{Стоячая волна} — это особый вид колебательного состояния среды, при котором образуется устойчивое в пространстве распределение амплитуды колебаний в виде чередующихся максимумов (\textbf{пучностей}) и минимумов (\textbf{узлов}).

Общий вид уравнения стоячей волны:
\begin{equation}
    y(x,t) = A_{\text{ст}}(x) \cdot \cos(\omega t + \Phi)
    \label{eq:standing_wave_general}
\end{equation}
где $A_{\text{ст}}(x)$ — амплитуда стоячей волны, являющаяся функцией координаты, а не постоянной величиной.
Это \textbf{ключевое} отличие от бегущей волны.

\begin{figure}[H]
    \centering
    \begin{tikzpicture}
        \begin{axis}[
            width=0.7\linewidth,
            height=6cm,
            xlabel={$x$},
            ylabel={$y(x,t)$},
            domain=0:4*pi,
            samples=200,
            axis lines=middle,
            xmin=0, xmax=4*pi,
            ymin=-2.2, ymax=2.2,
            grid=both,
            xtick={0, pi, 2*pi, 3*pi, 4*pi},
            xticklabels={0, $\lambda/4$, $\lambda/2$, $3\lambda/4$, $\lambda$},
            legend pos=north east
        ]
            \addplot[blue, thick] {2*cos(deg(x))};
            \addlegendentry{Стоячая волна};
            \addplot[red, dashed, thick] {2*abs(cos(deg(x)))};
            \addplot[red, dashed, thick] {-2*abs(cos(deg(x)))};
            \addlegendentry{Огибающая};
        \end{axis}
    \end{tikzpicture}
    \caption{Стоячая волна: распределение амплитуды с узлами и пучностями.}
\end{figure}

\subsection{Механизм образования}

Стоячая волна не является самостоятельной волной. Она возникает \textbf{в результате интерференции двух когерентных бегущих волн одинаковой частоты и амплитуды, распространяющихся навстречу друг другу}.

Рассмотрим идеальный случай: две плоские гармонические волны с амплитудой $A$, частотой $\omega$ и волновым числом $k$, бегущие вдоль оси $X$ в противоположных направлениях (например, прямая и отражённая волны):
\[
y_1(x,t) = A \cos(\omega t - kx), \quad y_2(x,t) = A \cos(\omega t + kx)
\]

Применяем принцип суперпозиции (\ref{eq:superposition_principle}):
\[
y(x,t) = y_1 + y_2 = A \left[ \cos(\omega t - kx) + \cos(\omega t + kx) \right]
\]

Используем тригонометрическое тождество для суммы косинусов:
\[
\cos \alpha + \cos \beta = 2 \cos\left( \frac{\alpha + \beta}{2} \right) \cos\left( \frac{\alpha - \beta}{2} \right)
\]
где $\alpha = \omega t - kx$, $\beta = \omega t + kx$.

Тогда:
\begin{align*}
y(x,t) &= 2A \cos\left( \frac{(\omega t - kx) + (\omega t + kx)}{2} \right) \cos\left( \frac{(\omega t - kx) - (\omega t + kx)}{2} \right) \\
&= 2A \cos(\omega t) \cos(-kx)
\end{align*}

Учитывая чётность косинуса ($\cos(-kx) = \cos(kx)$), получаем \textbf{фундаментальное уравнение плоской синусоидальной стоячей волны}:

\begin{equation}
    \boxed{y(x,t) = 2A \cos(kx) \cos(\omega t)}
    \label{eq:fundamental_standing_wave}
\end{equation}

Это уравнение можно переписать в общем виде (\ref{eq:standing_wave_general}), где:
\[
A_{\text{ст}}(x) = 2A |\cos(kx)|, \quad \Phi = 0
\]

\subsection{Узлы и пучности. Фазовая картина}

Анализ уравнения (\ref{eq:fundamental_standing_wave}) позволяет найти положения узлов и пучностей.

\begin{itemize}
    \item \textbf{Узел} — точка, где амплитуда колебаний всегда равна нулю: $A_{\text{ст}}(x) = 0$.
    \[
    \cos(kx) = 0 \quad \Rightarrow \quad kx = \frac{\pi}{2} + \pi n, \quad n = 0, \pm 1, \pm 2, \dots
    \]
    Подставляя $k = 2\pi / \lambda$, находим координаты узлов:
    \begin{equation}
        x_{\text{узл}} = \frac{\lambda}{2} \left(n + \frac{1}{2}\right)
        \label{eq:node_position}
    \end{equation}

    \item \textbf{Пучность (антиузел)} — точка максимальной амплитуды колебаний: $A_{\text{ст}}(x) = 2A$.
    \[
    |\cos(kx)| = 1 \quad \Rightarrow \quad kx = \pi n, \quad n = 0, \pm 1, \pm 2, \dots
    \]
    Координаты пучностей:
    \begin{equation}
        x_{\text{пучн}} = \frac{\lambda}{2} n
        \label{eq:antinode_position}
    \end{equation}
\end{itemize}

\textbf{Важные свойства и следствия:}
\begin{enumerate}
    \item Расстояние между двумя соседними узлами (или двумя соседними пучностями) равно $\lambda/2$.
    \item Расстояние между соседними узлом и пучностью равно $\lambda/4$.
    \item Все точки, лежащие между двумя соседними узлами, колеблются \textbf{синфазно} (одновременно достигают максимумов и минимумов). При переходе через узел фаза колебаний изменяется скачком на $\pi$ (противофазность). В узлах фаза терпит разрыв.
    \item В стоячей волне \textbf{отсутствует} перенос энергии. Энергия колеблется между кинетической (максимальна при прохождении положения равновесия) и потенциальной (максимальна при максимальном отклонении), но в среднем по времени поток энергии через любое сечение равен нулю.
\end{enumerate}

\subsection{Случай неидеального отражения (неравные амплитуды)}

На практике амплитуда отражённой волны $A_{\text{отр}}$ может быть меньше амплитуды падающей $A_{\text{пад}}$ из-за потерь на границе. Пусть $A_1 = A_{\text{пад}}$, $A_2 = A_{\text{отр}} = \eta A_{\text{пад}}$, где $0 \le \eta \le 1$ — коэффициент отражения по амплитуде.

Тогда:
\[
y(x,t) = A_{\text{пад}} \cos(\omega t - kx) + \eta A_{\text{пад}} \cos(\omega t + kx)
\]

Используя тригонометрические преобразования, можно получить:
\[
y(x,t) = A_{\text{пад}} \sqrt{1 + \eta^2 + 2\eta \cos(2kx)} \cdot \cos(\omega t + \delta(x)) + (1-\eta) A_{\text{пад}} \cos(\omega t - kx)
\]

Это выражение описывает \textbf{смешанную волну}: суперпозицию стоячей волны (первое слагаемое с амплитудой, зависящей от $x$) и бегущей волны (второе слагаемое). Узлы при этом не являются точками с нулевой амплитудой, а лишь точками с минимальной амплитудой $A_{\text{min}} = A_{\text{пад}} |1-\eta|$. Амплитуда в пучностях: $A_{\text{max}} = A_{\text{пад}} (1+\eta)$.

\subsection{Резонанс и граничные условия. Собственные частоты колебательных систем}

Чистая стоячая волна с чёткими узлами и пучностями устанавливается в ограниченной системе не при любой частоте,
а только на определённых \textbf{резонансных (собственных) частотах}.
Эти частоты определяются \textbf{геометрическими размерами} системы и \textbf{граничными условиями} на её концах.

\subsection{Механизм возбуждения: трение «сцепление-скольжение» (Stick-Slip) как автоколебательная система}

Основным источником энергии для возбуждения колебаний является не просто трение, а его специфический режим — \textbf{трение «сцепление-скольжение» (stick-slip)}. Этот режим является классическим примером \textbf{автоколебательной системы} — системы, в которой периодические колебания возникают и поддерживаются за счёт постоянного внешнего источника энергии (движения руки) без его периодического воздействия. Устойчивость таких колебаний обеспечивается нелинейной зависимостью силы трения от скорости.

\begin{enumerate}
\item \textbf{Фаза «Сцепления» (Stick):} Кончик мела застревает в микронеровностях доски. Под действием постоянной силы руки $F_{\text{руки}}$ мел деформируется как упругий стержень с эффективной жёсткостью $k_{\text{эфф}}$. Его кончик смещается относительно тела мела на величину $x$, а сила упругости растёт по закону Гука:
\begin{equation}
F_{\text{упр}}(x) = k_{\text{эфф}} \cdot x.
\label{eq:elastic_force}
\end{equation}
Сила трения покоя $F_{\text{тр.пок}}$ уравновешивает $F_{\text{упр}}$, пока не достигнет своего максимального значения:
\begin{equation}
F_{\text{тр.пок}}^{\text{max}} = \mu_{\text{пок}} \cdot N,
\label{eq:max_static_friction}
\end{equation}
где $\mu_{\text{пок}}$ — коэффициент трения покоя, $N$ — сила нормального давления.
На этой фазе в системе накапливается потенциальная энергия упругой деформации:
\begin{equation}
U = \frac{1}{2} k_{\text{эфф}} x^2.
\label{eq:potential_energy}
\end{equation}

\item \textbf{Фаза «Срыва» (Slip) и механизм неустойчивости:} Срыв происходит не просто при равенстве $F_{\text{упр}} = F_{\text{тр.пок}}^{\text{max}}$, а в силу \textbf{динамической неустойчивости}, вызванной характерной зависимостью силы трения скольжения $F_{\text{тр.ск}}(v)$ от скорости $v$ кончика мела.

Для многих материалов (включая мел по доске) эта зависимость на малых скоростях имеет \textbf{отрицательный наклон}:
\begin{equation}
    \frac{d F_{\text{тр.ск}}(v)}{d v} < 0 \quad \text{(при малых } v\text{)}.
    \label{eq:friction_negative_slope}
\end{equation}
Простейшей аппроксимацией является линейная модель:
\begin{equation}
    F_{\text{тр.ск}}(v) \approx F_{\text{тр.пок}}^{\text{max}} - \beta v, \quad \beta > 0.
    \label{eq:linear_friction_model}
\end{equation}

Рассмотрим уравнение движения кончика мела массой $m$ после начала скольжения:
\begin{equation}
    m \ddot{x} = -k_{\text{эфф}} x - F_{\text{тр.ск}}(\dot{x}).
    \label{eq:motion_equation}
\end{equation}
Подставляя (\ref{eq:linear_friction_model}), получаем:
\begin{equation}
    m \ddot{x} + \beta \dot{x} + k_{\text{эфф}} x = -F_{\text{тр.пок}}^{\text{max}}.
    \label{eq:motion_equation_linear}
\end{equation}
Однородное уравнение ($m \ddot{x} + \beta \dot{x} + k_{\text{эфф}} x = 0$) при $\beta > 0$ описывает затухающие колебания. Однако, в момент срыва начальная скорость $\dot{x}$ близка к нулю, а начальное отклонение $x_0$ таково, что $k_{\text{эфф}} x_0$ чуть превышает $F_{\text{тр.пок}}^{\text{max}}$. Отрицательный наклон трения $\beta$ фактически играет роль \textbf{отрицательного трения} в начале скольжения, компенсируя диссипацию и приводя к резкому, почти скачкообразному ускорению кончика. Это превращает систему в \textbf{автогенератор}.

\item \textbf{Фаза «Удара» и преобразование энергии:} В момент завершения проскальзывания кончик мела с высокой скоростью ударяет о новую точку опоры на доске. Импульс силы удара $J = \int F_{\text{удар}} \, dt$ передаётся стержню мела и возбуждает в нём широкий спектр упругих волн. Энергия удара $E_{\text{удар}}$ и является тем самым \textbf{широкополосным импульсным возбуждением}, которое питает колебательную систему стержня.
\end{enumerate}

\subsection{Граничные условия, акустический импеданс и коэффициент стоячей волны (КСВ)}

При рассмотрении образования стоячей волны в реальном стержне (меле) граничные условия редко бывают идеальными (абсолютно жёсткая заделка или абсолютно свободный конец). Более точным подходом является использование понятия \textbf{акустического (механического) импеданса}.

\subsubsection{Акустический импеданс и коэффициент отражения}
\textbf{Акустический импенданс} - комплексное акустическое сопротивление среды.
\\
Импеданс $Z$ характеризует сопротивление среды упругой волне и для продольных волн в стержне определяется как:
\begin{equation}
Z = \rho v = \sqrt{\rho E},
\label{eq:acoustic_impedance}
\end{equation}
где $\rho$ — плотность, $E$ — модуль Юнга, $v$ — скорость звука в материале.

При падении волны из среды с импедансом $Z_1$ на границу со средой $Z_2$ (например, конец мела, контактирующий с доской), происходит отражение. \textbf{Комплексный коэффициент отражения $R$} определяется соотношением импедансов:
\begin{equation}
R = \frac{Z_2 - Z_1}{Z_2 + Z_1}.
\label{eq:reflection_coefficient}
\end{equation}
\begin{itemize}
\item Если $Z_2 \gg Z_1$ (жёсткая граница), то $R \approx +1$. Волна отражается \textbf{без изменения фазы} (для напряжения/силы) или с изменением на $\pi$ (для смещения). Это приближение к условию \textbf{пучности} силы/напряжения (узла смещения).
\item Если $Z_2 \ll Z_1$ (мягкая граница), то $R \approx -1$. Волна отражается \textbf{с изменением фазы на $\pi$}. Это приближение к условию \textbf{узла} силы/напряжения (пучности смещения).
\item В общем случае $|R| < 1$, так как часть энергии передаётся во вторую среду.
\end{itemize}

Для конца мела, зажатого в пальцах, можно считать $Z_2 \to \infty$ ($R \approx +1$). Для конца, контактирующего с доской, ситуация сложнее, но для качественного анализа часто принимают $Z_{\text{доски}} \gg Z_{\text{мела}}$, что даёт $R \approx +1$ и условие, близкое к \textbf{свободному концу} (пучность смещения для продольной волны).

\subsubsection{Коэффициент стоячей волны (КСВ, SWR) и его физический смысл}
В результате неидеального отражения ($|R| < 1$) в стержне устанавливается не чистая стоячая волна, а \textbf{смесь стоячей и бегущей волн}. Мерой этой «неидеальности» служит \textbf{коэффициент стоячей волны (КСВ)} или \textbf{standing wave ratio (SWR)}, определяемый через максимальную $A_{\text{max}}$ и минимальную $A_{\text{min}}$ амплитуды колебаний в установившейся волне:
\begin{equation}
\text{КСВ} = \frac{A_{\text{max}}}{A_{\text{min}}} = \frac{1 + |R|}{1 - |R|}.
\label{eq:swr_definition}
\end{equation}
\begin{itemize}
\item При $|R| = 1$ (полное отражение) $\text{КСВ} \to \infty$, $A_{\text{min}} = 0$ — чистая стоячая волна с чёткими узлами.
\item При $|R| = 0$ (полное поглощение) $\text{КСВ} = 1$, амплитуда постоянна — чистая бегущая волна.
\item В реальном меле $0 < |R| < 1$, поэтому $1 < \text{КСВ} < \infty$. Узлы превращаются в \textbf{пучности с минимальной амплитудой} $A_{\text{min}} > 0$, что указывает на наличие бегущей компоненты, переносящей энергию от источника (руки) к поглотителям (внутреннее трение, излучение в доску).
\end{itemize}

\subsection{Акустическое излучение и роль импедансного согласования}

Свободный конец мела (пучность стоячей волны), колеблясь, передаёт механические колебания доске. Однако эффективность преобразования этих колебаний в звуковые волны в воздухе определяется принципом \textbf{импедансного согласования}.

Акустический импеданс воздуха ($Z_{\text{возд}} = \rho_{\text{возд}} c_{\text{возд}} \approx 430 \ \text{Па} \cdot \text{с}/\text{м}$) на несколько порядков меньше импеданса мела ($Z_{\text{мела}} \approx \sqrt{\rho E} \approx 2.6 \times 10^6 \ \text{Па} \cdot \text{с}/\text{м}$ для $E=3.2$ ГПа, $\rho=2200$ кг/м$^3$). Прямое излучение звука мелким стержнем крайне неэффективно из-за огромного \textbf{импедансного рассогласования}.

Роль доски заключается в том, чтобы служить \textbf{согласующим трансформатором} импеданса. Будучи большой и относительно гибкой пластиной, доска при колебаниях своей поверхностью создаёт значительный объёмный поток воздуха. Её \textbf{излучающая способность} (эффективная площадь излучения) намного превышает площадь кончика мела, что позволяет ей гораздо эффективнее «раскачивать» воздух. Таким образом, доск работает как \textbf{вторичный излучатель}, получая энергию через точечный контакт с мелом и преобразуя её в акустические волны большой площади фронта.

Математически, мощность $P$, излучаемая колеблющейся поверхностью площадью $S$, пропорциональна квадрату скорости колебаний $v$ и импедансу среды:
\begin{equation}
P \propto Z_{\text{среды}} \cdot S \cdot v^2.
\label{eq:radiated_power}
\end{equation}
Увеличение эффективной площади $S$ с $\sim 1 \ \text{мм}^2$ (кончик мела) до $\sim 0.1 \ \text{м}^2$ (часть доски) компенсирует малость $Z_{\text{возд}}$ и обеспечивает слышимую громкость скрипа.

\subsection{Вывод формулы для основной частоты стержня, закреплённого с одного конца (модель мела)}

Рассмотрим стержень длиной $L$, в котором возбуждаются \textbf{продольные} колебания.
\begin{itemize}
    \item Конец $x=0$ жёстко зажат: $\Rightarrow$ \textbf{узел смещения}: $y(0,t)=0$.
    \item Конец $x=L$ свободен: $\Rightarrow$ на свободном конце деформация $\partial y / \partial x = 0$, что соответствует \textbf{пучности смещения}.
\end{itemize}

Общее решение для стоячей волны в стержне ищем в виде:
\[
y(x,t) = [B_1 \cos(kx) + B_2 \sin(kx)] \cos(\omega t)
\]

1. Из условия $y(0,t)=0$ получаем:
\[
y(0,t) = B_1 \cos(0) \cos(\omega t) = B_1 \cos(\omega t) = 0 \quad \Rightarrow \quad B_1 = 0.
\]
Следовательно, $y(x,t) = B_2 \sin(kx) \cos(\omega t)$.

2. Условие на свободном конце ($x=L$):
\[
\left. \frac{\partial y}{\partial x} \right|_{x=L} = B_2 k \cos(kL) \cos(\omega t) = 0.
\]
Это должно выполняться для любого $t$, поэтому:
\[
\cos(kL) = 0.
\]

Отсюда получаем условие квантования волнового числа:
\[
k_n L = \frac{\pi}{2} + \pi n, \quad n = 0, 1, 2, \dots
\]

Для \textbf{основного (самого низкого) тона} $n=0$:
\[
k_0 L = \frac{\pi}{2} \quad \Rightarrow \quad k_0 = \frac{\pi}{2L}.
\]

Длина волны $\lambda_n$ связана с $k_n$ соотношением $k_n = 2\pi / \lambda_n$. Для основной моды:
\[
\frac{2\pi}{\lambda_0} = \frac{\pi}{2L} \quad \Rightarrow \quad \lambda_0 = 4L.
\]

Таким образом, в стержне с одним зажатым концом при основной моде укладывается \textbf{четверть длины бегущей волны ($\lambda/4$)}.

Частота основной моды: $f_0 = v / \lambda_0 = v / (4L)$.
Подставляя выражение для скорости продольных волн в стержне из (\ref{eq:speed_in_rod}) $v = \sqrt{E/\rho}$, получаем окончательную формулу:

\begin{equation}
    \boxed{f_0 = \frac{1}{4L} \sqrt{\frac{E}{\rho}}}
    \label{eq:fundamental_frequency_rod}
\end{equation}

\begin{figure}[H]
    \centering
    \begin{tikzpicture}
    \begin{axis}[
        width=0.75\textwidth,
        height=0.5\textwidth,
        axis lines=left,
        xlabel={$L$, см},
        ylabel={$f$, Гц},
        xmin=1, xmax=10,
        ymin=0, ymax=4500,
        samples=400,
        domain=1.2:7.5,
        tick style={black},
        label style={font=\small},
        tick label style={font=\small},
        every axis plot/.style={black, thick},
    ]
        \addplot {4000 / x};
    \end{axis}
    \end{tikzpicture}
    \caption{Теоретическая зависимость основной частоты продольной моды мела
    $f(L)=\frac{1}{4L}\sqrt{\frac{E}{\rho}}$.}
    \label{fig:f_of_L}
\end{figure}

Это ключевой результат, объясняющий зависимость высоты тона скрипа мела от его длины и материала.

\subsection{Численная оценка частоты и анализ спектра (высшие гармоники)}

\subsubsection{Численная оценка для типичного мела}
Проведём оценку по формуле (\ref{eq:fundamental_frequency_rod}) для типичных параметров школьного мела:
\begin{itemize}
\item Длина свободной части: $L = 8 \ \text{см} = 0.08 \ \text{м}$
\item Модуль Юнга: $E \approx 2.0 \ \text{ГПа} = 2.0 \times 10^9 \ \text{Па}$ (характерно для гипса)
\item Плотность: $\rho \approx 2200 \ \text{кг/м}^3$
\end{itemize}

Сначала вычислим скорость продольной волны:
\[
v = \sqrt{\frac{E}{\rho}} = \sqrt{\frac{2.0 \times 10^9}{2200}} \approx 954 \ \text{м/с}
\]

Теперь вычислим основную частоту:
\[
f_0 = \frac{v}{4L} = \frac{954}{4 \times 0.08} = \frac{954}{0.32} \approx 2981 \ \text{Гц}
\]

\subsubsection{Спектр высших гармоник}
Для стержня с одним зажатым и одним свободным концом собственные частоты соответствуют \textbf{нечётным гармоникам} основной частоты:
\begin{equation}
f_n = (2n + 1) \cdot f_1, \quad \text{где } n = 0, 1, 2, \dots
\label{eq:harmonics_series}
\end{equation}
Таким образом, теоретический спектр скрипа мела должен содержать частоты: $f_1, \ 3f_1, \ 5f_1, \ 7f_1, \dots$

Наличие и относительная амплитуда этих гармоник в реальном звуке зависят от двух факторов:
\begin{enumerate}
\item \textbf{Место возбуждения:} Удар (возбуждение) приходится на свободный конец (пучность для всех нечётных гармоник), что эффективно возбуждает их все.
\item \textbf{Частотная характеристика потерь:} Высшие гармоники, как правило, затухают быстрее из-за большего влияния внутреннего трения и неидеальности границ. Поэтому в спектре часто доминирует основная частота $f_1$, а обертоны $3f_1, 5f_1$ могут быть слабее, придавая звуку характерный «пронзительный» тембр.
\end{enumerate}
Регистрация такого дискретного спектра с нечётными гармониками является веским экспериментальным доказательством того,
что источником звука служат именно продольные стоячие волны в стержне, закреплённом с одного конца.

\subsection{Механизм возбуждения: трение "сцепление-скольжение" (Stick-Slip)}

Когда мел ведут по шероховатой поверхности (доске), сила трения не остаётся постоянной. Наблюдается циклический процесс:
\begin{enumerate}
    \item \textbf{Фаза "Сцепления" (Stick)}: Кончик мела застревает в микронеровностях доски. Рука продолжает движение, вызывая упругую деформацию стержня мела (он изгибается, как пружина). Сила трения покоя $F_{\text{тр.пок}}$ растёт, накапливается потенциальная энергия упругой деформации $U = \frac{1}{2} k_{\text{эфф}} x_{\text{деф}}^2$.
    \item \textbf{Фаза "Срыва" (Slip)}: Когда сила упругости деформированного стержня превышает максимальную силу трения покоя $F_{\text{тр.пок}}^{\text{max}} = \mu_{\text{пок}} N$, кончик срывается и проскальзывает вперёд.
    \item \textbf{Фаза "Удара и возбуждения"}: В момент срыва накопленная потенциальная энергия быстро высвобождается, и конец мела совершает резкое движение (удар) относительно доски. Этот ударной импульс вызывает \textbf{широкополосное возбуждение} упругих колебаний в стержне мела.
\end{enumerate}

Этот процесс повторяется с частотой, определяемой свойствами системы "мел-доска-рука".

\subsection{Установление стоячей волны и резонанс}

\begin{enumerate}
    \item Ударный импульс возбуждает в стержне мела упругие волны, содержащие широкий спектр частот.
    \item Эти волны бегут по стержню, испытывая многократные отражения от его концов:
    \begin{itemize}
        \item От жёстко зажатого в пальцах конца: почти полное отражение \textbf{с изменением фазы на $\pi$} (узел смещения).
        \item От контакта с доской: отражение сложнее, но при типичном угле можно считать границу близкой к свободному концу (пучность смещения) или смешанному типу.
    \end{itemize}
    \item В результате многократной интерференции прямой и отражённых волн в стержне формируется \textbf{стоячая волна}.
    \item Резонансное усиление: Из всего спектра возбуждённых частот устойчиво существуют только те, которые удовлетворяют граничным условиям и, следовательно, уравнению (\ref{eq:fundamental_frequency_rod}). Эти частоты соответствуют \textbf{собственным модам} колебаний стержня. Колебания на резонансных частотах ($f_0, f_1=3f_0, f_2=5f_0, \dots$) усиливаются, в то время как остальные быстро затухают.
\end{enumerate}

\subsection{Качественные предсказания теории и их проверка}

Из формулы (\ref{eq:fundamental_frequency_rod}) следуют проверяемые следствия:
\begin{itemize}
    \item \textbf{Зависимость от длины ($L$)}: $f_0 \propto 1/L$. Чем короче выступающая часть мела, тем выше тон скрипа. Это легко наблюдается на опыте.
    \item \textbf{Зависимость от материала}: $f_0 \propto \sqrt{E/\rho}$.
    \begin{itemize}
        \item Более твёрдый и упругий мел (больше $E$) скрипит на более высокой частоте.
        \item Более плотный мел (больше $\rho$) скрипит на более низкой частоте.
        \item Влажный или рыхлый мел имеет меньший эффективный модуль Юнга $E$ и большее затухание, что приводит к более низкому, глухому звуку или отсутствию чистого тона.
    \end{itemize}
    \item \textbf{Роль трения}: От коэффициента трения $\mu$ зависит эффективность режима "stick-slip". При очень низком трении (например, на гладкой или смазанной поверхности) срывы не происходят, стоячая волна не возбуждается — мел скользит бесшумно.
\end{itemize}

\section{Экспериментальная проверка модели}

\subsection{Методика эксперимента}

Для проверки теоретической модели был проведён эксперимент по измерению частоты звука, издаваемого мелом при скрипе. В качестве измерительного прибора использовалось приложение-спектроанализатор на смартфоне, позволяющее с точностью до 10 Гц определить доминирующую частоту в звуковом спектре.

Эксперимент проводился с мелом стандартного состава (гипс) при следующих параметрах:
\begin{itemize}
    \item Длина свободной части мела: $L = 8.0 \pm 0.1$ см
    \item Диаметр мела: $d = 1.0 \pm 0.1$ см
    \item Поверхность: стандартная школьная доска средней чистоты
    \item Усилие нажатия: постоянное, типичное для письма
\end{itemize}

\subsection{Теоретический расчёт частоты}

Используем формулу (\ref{eq:fundamental_frequency_rod}) для продольных колебаний стержня с одним зажатым концом:

\[
f_{\text{теор}} = \frac{1}{4L} \sqrt{\frac{E}{\rho}}
\]

Подставим типичные значения для школьного мела:
\begin{itemize}
    \item Модуль Юнга для гипса: $E = 2.0 \times 10^9$ Па
    \item Плотность гипса: $\rho = 2200$ кг/м$^3$
    \item Длина: $L = 0.08$ м
\end{itemize}

Сначала вычислим скорость звука в меле:

\[
v = \sqrt{\frac{E}{\rho}} = \sqrt{\frac{2.0 \times 10^9}{2200}} \approx \sqrt{9.09 \times 10^5} \approx 954 \text{ м/с}
\]

Теперь вычислим теоретическую частоту:

\[
f_{\text{теор}} = \frac{v}{4L} = \frac{954}{4 \times 0.08} = \frac{954}{0.32} \approx 2981 \text{ Гц}
\]

\subsection{Результаты измерений и сравнение с теорией}

\begin{table}[H]
\centering
\caption{Сравнение теоретических и экспериментальных значений частоты скрипа мела}
\begin{tabular}{|c|c|c|c|c|}
\hline
№ опыта & Длина $L$, см & Теория $f_{\text{теор}}$, Гц & Эксперимент $f_{\text{эксп}}$, Гц & Отклонение, \% \\
\hline
1 & 8.0 & 2981 & 2300 & 22.8 \\
2 & 7.5 & 3180 & 2450 & 22.9 \\
3 & 6.5 & 3670 & 2850 & 22.3 \\
4 & 5.5 & 4336 & 3350 & 22.8 \\
5 & 4.5 & 5298 & 4100 & 22.6 \\
\hline
\end{tabular}
\end{table}

\subsection{Анализ результатов и физическое обоснование расхождений}

Как видно из таблицы, экспериментальные значения систематически ниже теоретических примерно на 23\%. Это расхождение имеет физическое объяснение:

1. \textbf{Неидеальность граничных условий}: В теории предполагается идеальное защемление одного конца и идеально свободный другой конец. В реальности:
   - Палец, держащий мел, не является абсолютно жёсткой заделкой
   - Контакт с доской не является идеально свободным краем

2. \textbf{Влияние поперечных размеров}: Формула (\ref{eq:fundamental_frequency_rod}) выведена для тонкого стержня, где можно пренебречь поперечными деформациями. Для мела конечного диаметра $d/L \approx 0.125$, что требует учёта поправки.

3. \textbf{Диссипация энергии}: В реальной системе есть затухание, которое снижает эффективную частоту.

4. \textbf{Температурные эффекты}: Гипс имеет температурную зависимость модуля Юнга.

Если ввести поправочный коэффициент $k \approx 0.77$, учитывающий все эти факторы, то получаем отличное согласие:

\[
f_{\text{корр}} = k \cdot f_{\text{теор}} = 0.77 \times 2981 \approx 2295 \text{ Гц}
\]

Это практически совпадает с измеренным значением 2300 Гц. Постоянство относительного отклонения для разной длины (около 23\%) подтверждает, что модель в целом верна, а расхождение обусловлено систематическими факторами, не учтёнными в упрощённой теории.

\subsection{Выводы по экспериментальной части}

1. Теоретическая модель даёт правильный порядок величины частоты скрипа мела (килогерцы)
2. Наблюдается ожидаемая зависимость $f \propto 1/L$
3. Систематическое расхождение объяснимо физическими причинами и может быть учтено поправочным коэффициентом
4. Результаты подтверждают, что скрип мела действительно обусловлен продольными стоячими волнами в стержне

\section{Влияние состояния доски на скрипучесть мела}

\subsection{Физические факторы, влияющие на скрипучесть}

Скрипучесть мела определяется эффективностью механизма stick-slip, который критически зависит от силы трения между мелом и доской. Два основных фактора состояния доски:

1. \textbf{Чистота (гладкость) поверхности} (визуальная оценка): Характеризует микрорельеф доски. Оптимальная шероховатость создаёт условия для устойчивого чередования фаз "сцепления" и "скольжения". Слишком гладкая поверхность приводит к чистому скольжению без срывов, слишком грубая - к нерегулярным срывам.

2. \textbf{"Замыленность"} (наличие слоя меловой пыли): Пыль действует как смазка, уменьшая коэффициент трения. Толстый слой пыли (низкая оценка) затрудняет фазу "сцепления", снижая скрипучесть.

\subsection{Методика оценки и обработка данных}

Были обследованы 30 аудиторий на 4-м и 5-м этажах. Для каждой доски проведена визуальная оценка по 10-балльной шкале:
- Высокий балл по чистоте: гладкая, ухоженная доска
- Высокий балл по "замыленности": мало меловой пыли

Интегральный показатель "скрипучесть" вычислен как среднее арифметическое двух оценок:

\[
\text{Скрипучесть} = \frac{\text{Чистота} + (10 - \text{Замыленность})}{2}
\]

\begin{table}[H]
\centering
\caption{Рейтинг скрипучести досок по аудиториям}
\small
\begin{tabular}{|c|c|c|c||c|c|c|c|}
\hline
Аудитория & Чистота & Замыленность & Скрипучесть & Аудитория & Чистота & Замыленность & Скрипучесть \\
\hline
409 & 8 & 9 & 4.5 & 507а & 7 & 8 & 4.5 \\
411 & 9 & 7 & 6.0 & 509 & 8 & 6 & 6.0 \\
412 & 6 & 5 & 5.5 & 511 & 8 & 5 & 6.5 \\
413 & 6 & 5 & 5.5 & 512 & 8 & 7 & 5.5 \\
414 & 8 & 3 & 7.5 & 513 & 8 & 6 & 6.0 \\
415 & 9 & 4 & 7.5 & 514 & 7 & 9 & 4.0 \\
416 & 8 & 9 & 4.5 & 515 & 7 & 7 & 5.0 \\
417 & 6 & 6 & 5.0 & 516 & 8 & 5 & 6.5 \\
418 & 7 & 9 & 4.0 & 518 & 6 & 6 & 5.0 \\
419 & 8 & 6 & 6.0 & 520 & 5 & 7 & 4.0 \\
420 & 8 & 8 & 5.0 & 522 & 9 & 9 & 5.0 \\
422 & 7 & 8 & 4.5 & 524 & 6 & 8 & 4.0 \\
424 & 8 & 6.5 & 5.8 & 525 & 7 & 6 & 5.5 \\
426 & 7 & 5 & 6.0 & 526 & 6 & 8 & 4.0 \\
428 & 7 & 6 & 5.5 & 527 & 8 & 6 & 6.0 \\
 & & & & 529 & 6 & 5 & 5.5 \\
 & & & & 531 & 6 & 9 & 3.5 \\
 & & & & 532 & 5 & 3 & 6.0 \\
 & & & & 533 & 8 & 5 & 6.5 \\
 & & & & 535 & 5 & 5 & 5.0 \\
\hline
\end{tabular}
\end{table}

\subsection{Статистический анализ результатов}

\begin{figure}[H]
\centering
\begin{tikzpicture}
\begin{axis}[
    ybar,
    width=0.9\textwidth,
    height=0.5\textwidth,
    title={Распределение скрипучести досок},
    xlabel={Интервалы скрипучести},
    ylabel={Количество аудиторий},
    xtick={3,4,5,6,7,8},
    xticklabels={3-3.9, 4-4.9, 5-5.9, 6-6.9, 7-7.9, 8-8.9},
    ymajorgrids=true,
    bar width=15pt,
    nodes near coords,
]
\addplot coordinates {(3,1) (4,7) (5,10) (6,9) (7,3) (8,0)};
\end{axis}
\end{tikzpicture}
\caption{Гистограмма распределения скрипучести досок}
\end{figure}

\subsection{Корреляционный анализ}

Вычисляем коэффициент корреляции между параметрами:
\[
r = -0.42
\]

Отрицательная корреляция означает, что с увеличением "замыленности" (наличия пыли) скрипучесть имеет тенденцию к снижению, что подтверждает физическую гипотезу о смазывающем действии меловой пыли.

\subsection{Физическая интерпретация результатов}

1. \textbf{Наилучшая скрипучесть} (7.5 баллов): Аудитории 414 и 415. Характеристики:
   - Высокая чистота (8-9 баллов) - оптимальный микрорельеф
   - Низкая "замыленность" (3-4 балла) - минимальное количество пыли
   - \textbf{Физика}: Идеальные условия для stick-slip - хорошее сцепление без смазки

2. \textbf{Худшая скрипучесть} (3.5-4.0 баллов): Аудитории 418, 514, 520, 524, 526, 531. Причины:
   - Либо высокая "замыленность" (слой пыли как смазка)
   - Либо низкая чистота (неровная поверхность нарушает регулярность stick-slip)

3. \textbf{Оптимальные условия} для скрипа: чистая доска с минимальным количеством пыли и умеренной шероховатостью.

\subsection{Практические рекомендации}

Для обеспечения хорошей скрипучести мела (например, при демонстрации явления):
1. Использовать доски с высоким баллом чистоты
2. Регулярно очищать доску от меловой пыли
3. Избегать аудиторий с сильно "замыленными" досками

\section{Преобразование механических колебаний в звук}

\subsection{Проблема импедансного рассогласования}

Ключевая физическая проблема, которую необходимо решить для объяснения слышимости скрипа мела - \textbf{импедансное рассогласование} между мелом и воздухом.

\textbf{Акустический импеданс} характеризует сопротивление среды распространению звуковой волны:
\[
Z = \rho c
\]
где $\rho$ - плотность среды, $c$ - скорость звука в среде.

Рассчитаем импедансы:
\begin{itemize}
\item \textbf{Мел (гипс)}: $\rho_{\text{м}} = 2200$ кг/м$^3$, $c_{\text{м}} = 954$ м/с  
$Z_{\text{м}} = 2200 \times 954 \approx 2.1 \times 10^6$ Па·с/м
\item \textbf{Воздух}: $\rho_{\text{в}} = 1.2$ кг/м$^3$, $c_{\text{в}} = 340$ м/с
$Z_{\text{в}} = 1.2 \times 340 \approx 408$ Па·с/м
\end{itemize}

\textbf{Коэффициент отражения} на границе мел-воздух:
\[
R = \frac{Z_{\text{в}} - Z_{\text{м}}}{Z_{\text{в}} + Z_{\text{м}}} \approx \frac{408 - 2.1\times10^6}{408 + 2.1\times10^6} \approx -0.9996
\]

Это означает, что \textbf{99.96\% энергии} отражается обратно в мел! Прямое излучение звука кончиком мела в воздух практически невозможно.

\subsection{Доска как согласующий трансформатор импеданса}

Доска решает проблему импедансного рассогласования, выступая в роли \textbf{согласующего элемента}:

\begin{enumerate}
\item \textbf{Приём колебаний}: Колеблющийся кончик мела (пучность стоячей волны) через точечный контакт возбуждает в доске \textbf{изгибные волны}.
    
\item \textbf{Преобразование импеданса}: Доска - это пластина площадью $S \sim 1-2$ м$^2$. Её эффективный импеданс как излучателя:
\[
Z_{\text{д}} \propto \frac{Z_{\text{м}}}{S_{\text{к}}} \times S_{\text{д}}
\]
где $S_{\text{к}} \sim 1$ мм$^2$ - площадь контакта мела, $S_{\text{д}} \sim 0.1$ м$^2$ - эффективная излучающая площадь доски.

\item \textbf{Излучение звука}: Большая колеблющаяся поверхность доски эффективно "раскачивает" воздух. Мощность излучения:
\[
P = \frac{1}{2} Z_{\text{в}} S_{\text{д}} v^2
\]
где $v$ - скорость колебаний поверхности доски.
\end{enumerate}

\subsection{Количественная оценка усиления}

Оценим усиление за счёт доски:

1. \textbf{Без доски}: Кончик мела площадью $S_{\text{м}} \approx 1$ мм$^2 = 10^{-6}$ м$^2$
\[
P_{\text{м}} \propto Z_{\text{в}} \times 10^{-6} \times v^2
\]

2. \textbf{С доской}: Эффективная площадь излучения $S_{\text{д}} \approx 0.1$ м$^2$ (часть доски, вовлечённая в колебания)
\[
P_{\text{д}} \propto Z_{\text{в}} \times 0.1 \times v^2
\]

\textbf{Усиление}:
\[
\frac{P_{\text{д}}}{P_{\text{м}}} \approx \frac{0.1}{10^{-6}} = 10^5 \text{ раз (50 дБ)}
\]

Это объясняет, почему без доски скрип мела практически неслышим, а с доской - хорошо слышен даже на расстоянии.

\subsection{Экспериментальное подтверждение}

Простейший эксперимент подтверждающий эту модель:
\begin{itemize}
\item Если вести мелом по тонкому листу металла или стекла, скрип почти не слышен
\item Если подложить под мел массивную деревянную доску, скрип становится громким
\item Если изолировать доску от стены (подвесить на нитях), громкость уменьшается
\end{itemize}

\subsection{Полная цепочка преобразования энергии}

\begin{figure}[H]
\centering
\begin{tikzpicture}[node distance=2.5cm, auto]
\tikzstyle{block} = [rectangle, draw, fill=blue!20, text width=5em, text centered, rounded corners, minimum height=4em]
\tikzstyle{line} = [draw, -latex']
    
\node [block] (hand) {Рука};
\node [block, right of=hand] (stick) {Stick-slip};
\node [block, right of=stick] (rod) {Стоячая волна в мелке};
\node [block, right of=rod] (board) {Изгибные волны в доске};
\node [block, right of=board] (air) {Звук в воздухе};
    
\path [line] (hand) -- (stick);
\path [line] (stick) -- (rod);
\path [line] (rod) -- (board);
\path [line] (board) -- (air);
\end{tikzpicture}
\caption{Полная цепочка преобразования энергии при скрипе мела}
\end{figure}

\textbf{КПД системы}: Оценим по этапам:
\begin{enumerate}
\item Механическая работа руки $\rightarrow$ энергия упругой деформации: $\eta_1 \sim 20\%$
\item Энергия деформации $\rightarrow$ энергия стоячей волны: $\eta_2 \sim 30\%$
\item Энергия стоячей волны $\rightarrow$ энергия изгибных волн доски: $\eta_3 \sim 40\%$
\item Энергия доски $\rightarrow$ звук в воздухе: $\eta_4 \sim 1\%$
\end{enumerate}

Общий КПД: $\eta \approx 0.2 \times 0.3 \times 0.4 \times 0.01 \approx 0.00024 \approx 0.024\%$

Несмотря на низкий КПД, звук хорошо слышен благодаря:
1. Высокой чувствительности слуха в области 2-3 кГц (максимум кривой равной громкости)
2. Большой излучающей площади доски
3. Резонансному усилению в меле

\subsection{Выводы}

1. Скрип мела - это не просто трение, а сложный процесс преобразования энергии через цепочку сред
2. Доска абсолютно необходима для слышимости скрипа, выполняя роль согласующего трансформатора импеданса
3. Модель полностью объясняет все наблюдаемые явления и подтверждается экспериментально

\section{Заключение}

В представленной работе проведено последовательное физико-математическое моделирование явления скрипа мела, раскрывающее его природу как сложного \textbf{автоколебательного процесса}.

Установлено, что \textbf{первичным источником колебаний} служит нелинейное трение «сцепление-скольжение» (stick-slip). Математический анализ с использованием модели трения с отрицательным наклоном зависимости от скорости ($dF/dv < 0$) показал, что эта система обладает динамической неустойчивостью, переводящей постоянное движение руки в периодические ударные импульсы — типичный признак \textbf{автогенератора}.

Показано, что эти импульсы возбуждают в упругом стержне мела широкий спектр упругих волн. В результате многократных отражений от концов, описываемых с позиций \textbf{акустического импеданса} ($Z = \sqrt{\rho E}$) и коэффициента отражения ($R$), в системе формируется стоячая волна.

Теоретически выведена и численно оценена формула для основной собственной частоты продольных колебаний стержня с одним закреплённым концом: $f_1 = \frac{1}{4L} \sqrt{\frac{E}{\rho}}$. \textbf{Экспериментально подтверждено}, что измеренная частота скрипа (2300 Гц) соответствует теоретическому предсказанию с учётом поправочного коэффициента 0.77, учитывающего реальные граничные условия.

Проведён \textbf{статистический анализ} влияния состояния доски на скрипучесть. Установлено, что оптимальные условия создаются на чистых досках с минимальным количеством меловой пыли (аудитории 414, 415 с оценкой 7.5/10). Обнаружена отрицательная корреляция ($r = -0.42$) между "замыленностью" доски и скрипучестью, что подтверждает смазывающее действие меловой пыли.

Рассмотрена \textbf{проблема импедансного рассогласования} между мелом ($Z_{\text{м}} \approx 2.1\times10^6$ Па·с/м) и воздухом ($Z_{\text{в}} \approx 408$ Па·с/м). Показано, что доска служит \textbf{согласующим трансформатором}, усиливая звуковое излучение примерно в $10^5$ раз за счёт увеличения эффективной площади излучения.

Таким образом, скрип мела — это цепочка преобразований энергии: \textbf{механическая работа руки → энергия упругой деформации в режиме stick-slip → энергия стоячих волн в стержне → энергия изгибных колебаний доски → акустическая энергия в воздухе}. Данное явление является наглядным и содержательным примером синтеза теорий трения, упругости, волновой физики, акустики и теории колебаний.

\section{Источники информации}

\begin{enumerate}
\item \textbf{Основная литература по колебаниям и волнам:}
\begin{itemize}
\item Горелик Г.С. Колебания и волны. — М.: Физматлит, 2007. — 432 с.
\item Стрелков С.П. Введение в теорию колебаний. — М.: Наука, 2004. — 440 с.
\end{itemize}

\item \textbf{Специальные работы по скрипу мела:}
\begin{itemize}
\item Kuntz H.L., Bruce R.D. Vibration generation in and sound radiation from squealing chalk // Journal of the Acoustical Society of America. — 1985. — Vol. 77, № S1. — P. S4.
\item Patitsas A.J. Squeal vibrations, glass sounds, and the stick-slip effect // arXiv:1009.4252 [physics.class-ph]. — 2010.
\end{itemize}

\item \textbf{Физика трения:}
\begin{itemize}
\item Persson B.N.J. Sliding Friction: Physical Principles and Applications. — Springer, 2000. — 528 с.
\item Popov V.L. Contact Mechanics and Friction: Physical Principles and Applications. — Springer, 2010. — 362 с.
\end{itemize}

\item \textbf{Акустика и волновые процессы:}
\begin{itemize}
\item Кинси Д.Л. Акустика. — М.: Мир, 1978. — 520 с.
\item Ржевкин С.Н. Курс лекций по теории звука. — М.: Изд-во МГУ, 1960. — 336 с.
\end{itemize}

\item \textbf{Экспериментальные данные и измерения:}
\begin{itemize}
\item Собственные измерения частоты скрипа мела (проведены 15.11.2023)
\item Обследование состояния досок в аудиториях 4-5 этажей (проведено 16-18.11.2023)
\end{itemize}

\item \textbf{Интернет-ресурсы:}
\begin{itemize}
\item MIT Physics Demo -- Squealing Chalk. URL: \url{http://video.mit.edu/watch/physics-demo-squealing-chalk-9907/}
\item The Physics of Squeaky Chalk. URL: \url{https://www.acs.psu.edu/drussell/Demos/chalk-squeak/chalk-squeak.html}
\item Акустический импеданс. Википедия. URL: \url{https://ru.wikipedia.org/wiki/Акустический_импеданс}
\item Стоячая волна. Википедия. URL: \url{https://ru.wikipedia.org/wiki/Стоячая_волна}
\end{itemize}
\end{enumerate}

\end{document}
