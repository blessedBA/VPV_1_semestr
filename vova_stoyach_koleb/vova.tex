\documentclass[a4paper,12pt]{article}
\usepackage[utf8]{inputenc}
\usepackage[russian]{babel}
\usepackage{amsmath, amssymb, amsthm}
\usepackage{physics}
\usepackage{graphicx}
\usepackage{wrapfig}
\usepackage{siunitx}
\usepackage{booktabs}
\usepackage{caption}
\usepackage{subcaption}
\usepackage{float}
\usepackage{hyperref}
\usepackage{pgfplots}
\pgfplotsset{compat=1.18}

\title{\textbf{Физические основы стоячих волн и их проявление в акустических явлениях \\ (на примере скрипа мела)}}
\author{}
\date{}

\begin{document}

\maketitle

\tableofcontents

\newpage

\section{Введение}

Стоячая волна — фундаментальное физическое явление, возникающее при интерференции двух идентичных волн, распространяющихся навстречу друг другу. В отличие от бегущей волны, стоячая волна не переносит энергию в пространстве, а характеризуется устойчивым распределением амплитуды колебаний, состоящим из чередующихся максимумов (пучностей) и минимумов (узлов). Данная работа последовательно выводит математическую модель стоячей волны, исследует её свойства, устанавливает резонансные условия для её возникновения и применяет полученные результаты для объяснения механизма скрипа мела.

\section{Фундаментальные понятия: от колебаний к волнам}

\subsection{Гармонические колебания}

Любой волновой процесс основан на колебаниях. Простейшим и фундаментальным типом являются \textbf{гармонические колебания}, описываемые уравнением:

\begin{equation}
    y(t) = A \cos(\omega t + \varphi_0)
    \label{eq:simple_oscillation}
\end{equation}

где:
\begin{itemize}
    \item $y(t)$ — мгновенное значение колеблющейся величины (смещение, давление),
    \item $A$ — \textbf{амплитуда} колебаний (максимальное отклонение),
    \item $\omega = 2\pi f$ — \textbf{циклическая (круговая) частота},
    \item $f$ — частота колебаний в Герцах (Гц),
    \item $\varphi_0$ — \textbf{начальная фаза}.
\end{itemize}

Период колебаний $T$, за который фаза изменяется на $2\pi$, связан с частотой: $T = 1/f = 2\pi/\omega$.

\subsection{Бегущая волна как перенос колебаний}

Если гармоническое колебание, возникнув в одной точке упругой среды (воздух, твёрдое тело), передаётся соседним точкам, возникает \textbf{волна} — процесс распространения колебаний в пространстве с переносом энергии, но без переноса вещества.

Для \textbf{плоской гармонической бегущей волны}, распространяющейся вдоль положительного направления оси $x$, уравнение имеет вид:

\begin{equation}
    y(x, t) = A \cos(\omega t - kx + \varphi_0)
    \label{eq:traveling_wave_plus}
\end{equation}

Для волны, бегущей в противоположном направлении:

\begin{equation}
    y(x, t) = A \cos(\omega t + kx + \varphi_0)
    \label{eq:traveling_wave_minus}
\end{equation}

Здесь введён новый параметр:
\begin{itemize}
    \item $k = \dfrac{2\pi}{\lambda}$ — \textbf{волновое число}, где $\lambda$ — \textbf{длина волны} (минимальное расстояние между точками, колеблющимися в одинаковой фазе).
\end{itemize}

Аргумент косинуса $(\omega t \mp kx + \varphi_0)$ называется \textbf{фазой волны}. Постоянство фазы ($\omega t \mp kx = \text{const}$) определяет движение волнового фронта. Дифференцируя это условие по времени, получаем \textbf{фазовую скорость волны} $v$:

\[
\omega \, dt \mp k \, dx = 0 \quad \Rightarrow \quad \frac{dx}{dt} = \pm \frac{\omega}{k}
\]

\begin{equation}
    v = \frac{\omega}{k} = \lambda f
    \label{eq:wave_velocity}
\end{equation}

Скорость распространения волны зависит от свойств среды. Для \textbf{продольных упругих волн} в тонком стержне она определяется формулой:

\begin{equation}
    v = \sqrt{\frac{E}{\rho}}
    \label{eq:speed_in_rod}
\end{equation}
где:
\begin{itemize}
    \item $E$ — модуль Юнга материала (характеризует упругость),
    \item $\rho$ — плотность материала.
\end{itemize}

\section{Принцип суперпозиции и интерференция}

\subsection{Принцип суперпозиции}

\textbf{Принцип суперпозиции (наложения)} является фундаментальным для линейных волновых процессов: если в среде распространяется несколько волн, то результирующее смещение любой частицы среды в любой момент времени равно векторной сумме смещений, которые вызывала бы каждая из волн в отдельности.

Математически для двух волн $y_1(x,t)$ и $y_2(x,t)$:
\begin{equation}
    y_{\text{рез}}(x,t) = y_1(x,t) + y_2(x,t)
    \label{eq:superposition_principle}
\end{equation}

\subsection{Интерференция}

Интерференция — это частное следствие принципа суперпозиции, проявляющееся в образовании устойчивой картины усиления и ослабления колебаний в разных точках пространства при сложении \textbf{когерентных волн}. Волны называются когерентными, если они имеют \textbf{одинаковую частоту} и \textbf{постоянную во времени разность фаз}.

Результат интерференции зависит от разности фаз $\Delta \varphi$ складываемых волн в точке наблюдения:
\begin{itemize}
    \item \textbf{Усиление (максимум)}: $\Delta \varphi = 2\pi m$, где $m = 0, \pm 1, \pm 2, \dots$
    \item \textbf{Ослабление (минимум)}: $\Delta \varphi = \pi (2m + 1)$.
\end{itemize}

Для двух волн, приходящих в точку от разных источников, разность фаз определяется разностью расстояний до источников (\textbf{разностью хода} $\Delta L$): $\Delta \varphi = k \Delta L = \frac{2\pi}{\lambda} \Delta L$.

\section{Стоячая волна: определение, механизм образования и свойства}

\subsection{Определение и общий вид}

\textbf{Стоячая волна} — это особый вид колебательного состояния среды, при котором образуется устойчивое в пространстве распределение амплитуды колебаний в виде чередующихся максимумов (\textbf{пучностей}) и минимумов (\textbf{узлов}).

Общий вид уравнения стоячей волны:
\begin{equation}
    y(x,t) = A_{\text{ст}}(x) \cdot \cos(\omega t + \Phi)
    \label{eq:standing_wave_general}
\end{equation}
где $A_{\text{ст}}(x)$ — амплитуда стоячей волны, являющаяся функцией координаты, а не постоянной величиной. Это ключевое отличие от бегущей волны.

\subsection{Механизм образования и математический вывод}

Стоячая волна не является самостоятельной волной. Она возникает \textbf{в результате интерференции двух когерентных бегущих волн одинаковой частоты и амплитуды, распространяющихся навстречу друг другу}.

Рассмотрим идеальный случай: две плоские гармонические волны с амплитудой $A$, частотой $\omega$ и волновым числом $k$, бегущие вдоль оси $X$ в противоположных направлениях (например, прямая и отражённая волны):
\[
y_1(x,t) = A \cos(\omega t - kx), \quad y_2(x,t) = A \cos(\omega t + kx)
\]

Применяем принцип суперпозиции (\ref{eq:superposition_principle}):
\[
y(x,t) = y_1 + y_2 = A \left[ \cos(\omega t - kx) + \cos(\omega t + kx) \right]
\]

Используем тригонометрическое тождество для суммы косинусов:
\[
\cos \alpha + \cos \beta = 2 \cos\left( \frac{\alpha + \beta}{2} \right) \cos\left( \frac{\alpha - \beta}{2} \right)
\]
где $\alpha = \omega t - kx$, $\beta = \omega t + kx$.

Тогда:
\begin{align*}
y(x,t) &= 2A \cos\left( \frac{(\omega t - kx) + (\omega t + kx)}{2} \right) \cos\left( \frac{(\omega t - kx) - (\omega t + kx)}{2} \right) \\
&= 2A \cos(\omega t) \cos(-kx)
\end{align*}

Учитывая чётность косинуса ($\cos(-kx) = \cos(kx)$), получаем \textbf{фундаментальное уравнение плоской синусоидальной стоячей волны}:

\begin{equation}
    \boxed{y(x,t) = 2A \cos(kx) \cos(\omega t)}
    \label{eq:fundamental_standing_wave}
\end{equation}

Это уравнение можно переписать в общем виде (\ref{eq:standing_wave_general}), где:
\[
A_{\text{ст}}(x) = 2A |\cos(kx)|, \quad \Phi = 0
\]

\subsection{Узлы и пучности. Фазовая картина}

Анализ уравнения (\ref{eq:fundamental_standing_wave}) позволяет найти положения узлов и пучностей.

\begin{itemize}
    \item \textbf{Узел} — точка, где амплитуда колебаний всегда равна нулю: $A_{\text{ст}}(x) = 0$.
    \[
    \cos(kx) = 0 \quad \Rightarrow \quad kx = \frac{\pi}{2} + \pi n, \quad n = 0, \pm 1, \pm 2, \dots
    \]
    Подставляя $k = 2\pi / \lambda$, находим координаты узлов:
    \begin{equation}
        x_{\text{узл}} = \frac{\lambda}{2} \left(n + \frac{1}{2}\right)
        \label{eq:node_position}
    \end{equation}

    \item \textbf{Пучность (антиузел)} — точка максимальной амплитуды колебаний: $A_{\text{ст}}(x) = 2A$.
    \[
    |\cos(kx)| = 1 \quad \Rightarrow \quad kx = \pi n, \quad n = 0, \pm 1, \pm 2, \dots
    \]
    Координаты пучностей:
    \begin{equation}
        x_{\text{пучн}} = \frac{\lambda}{2} n
        \label{eq:antinode_position}
    \end{equation}
\end{itemize}

\textbf{Важные свойства и следствия:}
\begin{enumerate}
    \item Расстояние между двумя соседними узлами (или двумя соседними пучностями) равно $\lambda/2$.
    \item Расстояние между соседними узлом и пучностью равно $\lambda/4$.
    \item Все точки, лежащие между двумя соседними узлами, колеблются \textbf{синфазно} (одновременно достигают максимумов и минимумов). При переходе через узел фаза колебаний изменяется скачком на $\pi$ (противофазность). В узлах фаза терпит разрыв.
    \item В стоячей волне отсутствует перенос энергии. Энергия колеблется между кинетической (максимальна при прохождении положения равновесия) и потенциальной (максимальна при максимальном отклонении), но в среднем по времени поток энергии через любое сечение равен нулю.
\end{enumerate}

\subsection{Случай неидеального отражения (неравные амплитуды)}

На практике амплитуда отражённой волны $A_{\text{отр}}$ может быть меньше амплитуды падающей $A_{\text{пад}}$ из-за потерь на границе. Пусть $A_1 = A_{\text{пад}}$, $A_2 = A_{\text{отр}} = \eta A_{\text{пад}}$, где $0 \le \eta \le 1$ — коэффициент отражения по амплитуде.

Тогда:
\[
y(x,t) = A_{\text{пад}} \cos(\omega t - kx) + \eta A_{\text{пад}} \cos(\omega t + kx)
\]

Используя тригонометрические преобразования, можно получить:
\[
y(x,t) = A_{\text{пад}} \sqrt{1 + \eta^2 + 2\eta \cos(2kx)} \cdot \cos(\omega t + \delta(x)) + (1-\eta) A_{\text{пад}} \cos(\omega t - kx)
\]

Это выражение описывает \textbf{смешанную волну}: суперпозицию стоячей волны (первое слагаемое с амплитудой, зависящей от $x$) и бегущей волны (второе слагаемое). Узлы при этом не являются точками с нулевой амплитудой, а лишь точками с минимальной амплитудой $A_{\text{min}} = A_{\text{пад}} |1-\eta|$. Амплитуда в пучностях: $A_{\text{max}} = A_{\text{пад}} (1+\eta)$.

\section{Резонанс и граничные условия. Собственные частоты колебательных систем}

Чистая стоячая волна с чёткими узлами и пучностями устанавливается в ограниченной системе не при любой частоте, а только на определённых \textbf{резонансных (собственных) частотах}. Эти частоты определяются \textbf{геометрическими размерами} системы и \textbf{граничными условиями} на её концах.

\subsection{Механизм возбуждения: трение "сцепление-\\\скольжение" (Stick-Slip) как автоколебательная система}

Основным источником энергии для возбуждения колебаний является не просто трение, а его специфический режим — \textbf{трение «сцепление-скольжение» (stick-slip)}. Этот режим является классическим примером \textbf{автоколебательной системы} — системы, в которой периодические колебания возникают и поддерживаются за счёт постоянного внешнего источника энергии (движения руки) без его периодического воздействия. Устойчивость таких колебаний обеспечивается нелинейной зависимостью силы трения от скорости.

\begin{enumerate}
\item \textbf{Фаза «Сцепления» (Stick):} Кончик мела застревает в микронеровностях доски. Под действием постоянной силы руки $F_{\text{руки}}$ мел деформируется как упругий стержень с эффективной жёсткостью $k_{\text{эфф}}$. Его кончик смещается относительно тела мела на величину $x$, а сила упругости растёт по закону Гука:
\begin{equation}
F_{\text{упр}}(x) = k_{\text{эфф}} \cdot x.
\label{eq:elastic_force}
\end{equation}
Сила трения покоя $F_{\text{тр.пок}}$ уравновешивает $F_{\text{упр}}$, пока не достигнет своего максимального значения:
\begin{equation}
F_{\text{тр.пок}}^{\text{max}} = \mu_{\text{пок}} \cdot N,
\label{eq:max_static_friction}
\end{equation}
где $\mu_{\text{пок}}$ — коэффициент трения покоя, $N$ — сила нормального давления.
На этой фазе в системе накапливается потенциальная энергия упругой деформации:
\begin{equation}
U = \frac{1}{2} k_{\text{эфф}} x^2.
\label{eq:potential_energy}
\end{equation}

text
\item \textbf{Фаза «Срыва» (Slip) и механизм неустойчивости:} Срыв происходит не просто при равенстве $F_{\text{упр}} = F_{\text{тр.пок}}^{\text{max}}$, а в силу \textbf{динамической неустойчивости}, вызванной характерной зависимостью силы трения скольжения $F_{\text{тр.ск}}(v)$ от скорости $v$ кончика мела.

Для многих материалов (включая мел по доске) эта зависимость на малых скоростях имеет \textbf{отрицательный наклон}:
\begin{equation}
    \frac{d F_{\text{тр.ск}}(v)}{d v} < 0 \quad \text{(при малых } v\text{)}.
    \label{eq:friction_negative_slope}
\end{equation}
Простейшей аппроксимацией является линейная модель:
\begin{equation}
    F_{\text{тр.ск}}(v) \approx F_{\text{тр.пок}}^{\text{max}} - \beta v, \quad \beta > 0.
    \label{eq:linear_friction_model}
\end{equation}

Рассмотрим уравнение движения кончика мела массой $m$ после начала скольжения:
\begin{equation}
    m \ddot{x} = -k_{\text{эфф}} x - F_{\text{тр.ск}}(\dot{x}).
    \label{eq:motion_equation}
\end{equation}
Подставляя (\ref{eq:linear_friction_model}), получаем:
\begin{equation}
    m \ddot{x} + \beta \dot{x} + k_{\text{эфф}} x = -F_{\text{тр.пок}}^{\text{max}}.
    \label{eq:motion_equation_linear}
\end{equation}
Однородное уравнение ($m \ddot{x} + \beta \dot{x} + k_{\text{эфф}} x = 0$) при $\beta > 0$ описывает затухающие колебания. Однако, в момент срыва начальная скорость $\dot{x}$ близка к нулю, а начальное отклонение $x_0$ таково, что $k_{\text{эфф}} x_0$ чуть превышает $F_{\text{тр.пок}}^{\text{max}}$. Отрицательный наклон трения $\beta$ фактически играет роль \textbf{отрицательного трения} в начале скольжения, компенсируя диссипацию и приводя к резкому, почти скачкообразному ускорению кончика. Это превращает систему в \textbf{автогенератор}.

\item \textbf{Фаза «Удара» и преобразование энергии:} В момент завершения проскальзывания кончик мела с высокой скоростью ударяет о новую точку опоры на доске. Импульс силы удара $J = \int F_{\text{удар}} \, dt$ передаётся стержню мела и возбуждает в нём широкий спектр упругих волн. Энергия удара $E_{\text{удар}}$ и является тем самым \textbf{широкополосным импульсным возбуждением}, которое питает колебательную систему стержня.
\end{enumerate}

\subsection{Граничные условия, акустический импеданс и коэффициент стоячей волны (КСВ)}

При рассмотрении образования стоячей волны в реальном стержне (меле) граничные условия редко бывают идеальными (абсолютно жёсткая заделка или абсолютно свободный конец). Более точным подходом является использование понятия \textbf{акустического (механического) импеданса}.

\subsubsection*{Акустический импеданс и коэффициент отражения}
Импеданс $Z$ характеризует сопротивление среды упругой волне и для продольных волн в стержне определяется как:
\begin{equation}
Z = \rho v = \sqrt{\rho E},
\label{eq:acoustic_impedance}
\end{equation}
где $\rho$ — плотность, $E$ — модуль Юнга, $v$ — скорость звука в материале.

При падении волны из среды с импедансом $Z_1$ на границу со средой $Z_2$ (например, конец мела, контактирующий с доской), происходит отражение. \textbf{Комплексный коэффициент отражения $R$} определяется соотношением импедансов:
\begin{equation}
R = \frac{Z_2 - Z_1}{Z_2 + Z_1}.
\label{eq:reflection_coefficient}
\end{equation}
\begin{itemize}
\item Если $Z_2 \gg Z_1$ (жёсткая граница), то $R \approx +1$. Волна отражается \textbf{без изменения фазы} (для напряжения/силы) или с изменением на $\pi$ (для смещения). Это приближение к условию \textbf{пучности} силы/напряжения (узла смещения).
\item Если $Z_2 \ll Z_1$ (мягкая граница), то $R \approx -1$. Волна отражается \textbf{с изменением фазы на $\pi$}. Это приближение к условию \textbf{узла} силы/напряжения (пучности смещения).
\item В общем случае $|R| < 1$, так как часть энергии передаётся во вторую среду.
\end{itemize}

Для конца мела, зажатого в пальцах, можно считать $Z_2 \to \infty$ ($R \approx +1$). Для конца, контактирующего с доской, ситуация сложнее, но для качественного анализа часто принимают $Z_{\text{доски}} \gg Z_{\text{мела}}$, что даёт $R \approx +1$ и условие, близкое к \textbf{свободному концу} (пучность смещения для продольной волны).

\subsubsection*{Коэффициент стоячей волны (КСВ, SWR) и его физический смысл}
В результате неидеального отражения ($|R| < 1$) в стержне устанавливается не чистая стоячая волна, а \textbf{смесь стоячей и бегущей волн}. Мерой этой «неидеальности» служит \textbf{коэффициент стоячей волны (КСВ)} или \textbf{standing wave ratio (SWR)}, определяемый через максимальную $A_{\text{max}}$ и минимальную $A_{\text{min}}$ амплитуды колебаний в установившейся волне:
\begin{equation}
\text{КСВ} = \frac{A_{\text{max}}}{A_{\text{min}}} = \frac{1 + |R|}{1 - |R|}.
\label{eq:swr_definition}
\end{equation}
\begin{itemize}
\item При $|R| = 1$ (полное отражение) $\text{КСВ} \to \infty$, $A_{\text{min}} = 0$ — чистая стоячая волна с чёткими узлами.
\item При $|R| = 0$ (полное поглощение) $\text{КСВ} = 1$, амплитуда постоянна — чистая бегущая волна.
\item В реальном меле $0 < |R| < 1$, поэтому $1 < \text{КСВ} < \infty$. Узлы превращаются в \textbf{пучности с минимальной амплитудой} $A_{\text{min}} > 0$, что указывает на наличие бегущей компоненты, переносящей энергию от источника (руки) к поглотителям (внутреннее трение, излучение в доску).
\end{itemize}

\subsection{Акустическое излучение и роль импедансного согласования}

Свободный конец мела (пучность стоячей волны), колеблясь, передаёт механические колебания доске. Однако эффективность преобразования этих колебаний в звуковые волны в воздухе определяется принципом \textbf{импедансного согласования}.

Акустический импеданс воздуха ($Z_{\text{возд}} = \rho_{\text{возд}} c_{\text{возд}} \approx 430 \ \text{Па} \cdot \text{с}/\text{м}$) на несколько порядков меньше импеданса мела ($Z_{\text{мела}} \approx \sqrt{\rho E} \approx 2.6 \times 10^6 \ \text{Па} \cdot \text{с}/\text{м}$ для $E=3.2$ ГПа, $\rho=2200$ кг/м$^3$). Прямое излучение звука мелким стержнем крайне неэффективно из-за огромного \textbf{импедансного рассогласования}.

Роль доски заключается в том, чтобы служить \textbf{согласующим трансформатором} импеданса. Будучи большой и относительно гибкой пластиной, доска при колебаниях своей поверхностью создаёт значительный объёмный поток воздуха. Её \textbf{излучающая способность} (эффективная площадь излучения) намного превышает площадь кончика мела, что позволяет ей гораздо эффективнее «раскачивать» воздух. Таким образом, доска работает как \textbf{вторичный излучатель}, получая энергию через точечный контакт с мелом и преобразуя её в акустические волны большой площади фронта.

Математически, мощность $P$, излучаемая колеблющейся поверхностью площадью $S$, пропорциональна квадрату скорости колебаний $v$ и импедансу среды:
\begin{equation}
P \propto Z_{\text{среды}} \cdot S \cdot v^2.
\label{eq:radiated_power}
\end{equation}
Увеличение эффективной площади $S$ с $\sim 1 \ \text{мм}^2$ (кончик мела) до $\sim 0.1 \ \text{м}^2$ (часть доски) компенсирует малость $Z_{\text{возд}}$ и обеспечивает слышимую громкость скрипа.

Здесь $v$ — скорость волны в соответствующей среде.

\subsection{Вывод формулы для основной частоты стержня, закреплённого с одного конца (модель мела)}

Рассмотрим стержень длиной $L$, в котором возбуждаются \textbf{продольные} колебания.
\begin{itemize}
    \item Конец $x=0$ жёстко зажат: $\Rightarrow$ \textbf{узел смещения}: $y(0,t)=0$.
    \item Конец $x=L$ свободен: $\Rightarrow$ на свободном конце деформация $\partial y / \partial x = 0$, что соответствует \textbf{пучности смещения}.
\end{itemize}

Общее решение для стоячей волны в стержне ищем в виде:
\[
y(x,t) = [B_1 \cos(kx) + B_2 \sin(kx)] \cos(\omega t)
\]

1. Из условия $y(0,t)=0$ получаем:
\[
y(0,t) = B_1 \cos(0) \cos(\omega t) = B_1 \cos(\omega t) = 0 \quad \Rightarrow \quad B_1 = 0.
\]
Следовательно, $y(x,t) = B_2 \sin(kx) \cos(\omega t)$.

2. Условие на свободном конце ($x=L$):
\[
\left. \frac{\partial y}{\partial x} \right|_{x=L} = B_2 k \cos(kL) \cos(\omega t) = 0.
\]
Это должно выполняться для любого $t$, поэтому:
\[
\cos(kL) = 0.
\]

Отсюда получаем условие квантования волнового числа:
\[
k_n L = \frac{\pi}{2} + \pi n, \quad n = 0, 1, 2, \dots
\]

Для \textbf{основного (самого низкого) тона} $n=0$:
\[
k_0 L = \frac{\pi}{2} \quad \Rightarrow \quad k_0 = \frac{\pi}{2L}.
\]

Длина волны $\lambda_n$ связана с $k_n$ соотношением $k_n = 2\pi / \lambda_n$. Для основной моды:
\[
\frac{2\pi}{\lambda_0} = \frac{\pi}{2L} \quad \Rightarrow \quad \lambda_0 = 4L.
\]

Таким образом, в стержне с одним зажатым концом при основной моде укладывается \textbf{четверть длины бегущей волны ($\lambda/4$)}.

Частота основной моды: $f_0 = v / \lambda_0 = v / (4L)$.
Подставляя выражение для скорости продольных волн в стержне из (\ref{eq:speed_in_rod}) $v = \sqrt{E/\rho}$, получаем окончательную формулу:

\begin{equation}
    \boxed{f_0 = \frac{1}{4L} \sqrt{\frac{E}{\rho}}}
    \label{eq:fundamental_frequency_rod}
\end{equation}

Это ключевой результат, объясняющий зависимость высоты тона скрипа мела от его длины и материала.

\subsection{Численная оценка частоты и анализ спектра (высшие гармоники)}

\subsubsection*{Численная оценка для типичного мела}
Проведём оценку по формуле (\ref{eq:fundamental_frequency_rod}) для типичных параметров школьного мела:
\begin{itemize}
\item Длина свободной части: $L = 5 \ \text{см} = 0.05 \ \text{м}$
\item Модуль Юнга: $E \approx 3.2 \ \text{ГПа} = 3.2 \times 10^9 \ \text{Па}$ (характерно для гипса)
\item Плотность: $\rho \approx 2200 \ \text{кг/м}^3$
\end{itemize}
Скорость продольной волны

\subsubsection*{Спектр высших гармоник}
Для стержня с одним зажатым и одним свободным концом собственные частоты соответствуют \textbf{нечётным гармоникам} основной частоты:
\begin{equation}
f_n = (2n + 1) \cdot f_1, \quad \text{где } n = 0, 1, 2, \dots
\label{eq:harmonics_series}
\end{equation}
Таким образом, теоретический спектр скрипа мела должен содержать частоты: $f_1, \ 3f_1, \ 5f_1, \ 7f_1, \dots$

Наличие и относительная амплитуда этих гармоник в реальном звуке зависят от двух факторов:
\begin{enumerate}
\item \textbf{Место возбуждения:} Удар (возбуждение) приходится на свободный конец (пучность для всех нечётных гармоник), что эффективно возбуждает их все.
\item \textbf{Частотная характеристика потерь:} Высшие гармоники, как правило, затухают быстрее из-за большего влияния внутреннего трения и неидеальности границ. Поэтому в спектре часто доминирует основная частота $f_1$, а обертоны $3f_1, 5f_1$ могут быть слабее, придавая звуку характерный «пронзительный» тембр.
\end{enumerate}
Регистрация такого дискретного спектра с нечётными гармониками является веским экспериментальным доказательством того, что источником звука служат именно продольные стоячие волны в стержне, закреплённом с одного конца.

\section{Скрип мела: синтез теории}

\subsection{Механизм возбуждения: трение "сцепление-скольжение" (Stick-Slip)}

Когда мел ведут по шероховатой поверхности (доске), сила трения не остаётся постоянной. Наблюдается циклический процесс:
\begin{enumerate}
    \item \textbf{Фаза "Сцепления" (Stick)}: Кончик мела застревает в микронеровностях доски. Рука продолжает движение, вызывая упругую деформацию стержня мела (он изгибается, как пружина). Сила трения покоя $F_{\text{тр.пок}}$ растёт, накапливается потенциальная энергия упругой деформации $U = \frac{1}{2} k_{\text{эфф}} x_{\text{деф}}^2$.
    \item \textbf{Фаза "Срыва" (Slip)}: Когда сила упругости деформированного стержня превышает максимальную силу трения покоя $F_{\text{тр.пок}}^{\text{max}} = \mu_{\text{пок}} N$, кончик срывается и проскальзывает вперёд.
    \item \textbf{Фаза "Удара и возбуждения"}: В момент срыва накопленная потенциальная энергия быстро высвобождается, и конец мела совершает резкое движение (удар) относительно доски. Этот ударной импульс вызывает \textbf{широкополосное возбуждение} упругих колебаний в стержне мела.
\end{enumerate}

Этот процесс повторяется с частотой, определяемой свойствами системы "мел-доска-рука".

\subsection{Установление стоячей волны и резонанс}

\begin{enumerate}
    \item Ударный импульс возбуждает в стержне мела упругие волны, содержащие широкий спектр частот.
    \item Эти волны бегут по стержню, испытывая многократные отражения от его концов:
    \begin{itemize}
        \item От жёстко зажатого в пальцах конца: почти полное отражение \textbf{с изменением фазы на $\pi$} (узел смещения).
        \item От контакта с доской: отражение сложнее, но при типичном угле можно считать границу близкой к свободному концу (пучность смещения) или смешанному типу.
    \end{itemize}
    \item В результате многократной интерференции прямой и отражённых волн в стержне формируется \textbf{стоячая волна}.
    \item Резонансное усиление: Из всего спектра возбуждённых частот устойчиво существуют только те, которые удовлетворяют граничным условиям и, следовательно, уравнению (\ref{eq:fundamental_frequency_rod}). Эти частоты соответствуют \textbf{собственным модам} колебаний стержня. Колебания на резонансных частотах ($f_0, f_1=3f_0, f_2=5f_0, \dots$) усиливаются, в то время как остальные быстро затухают.
\end{enumerate}

\subsection{Акустическое излучение и восприятие звука}

Свободный конец мела (пучность стоячей волны) колеблется с большой амплитудой и периодически ударяет по поверхности доски. Доска, обладающая большой площадью, работает как эффективный \textbf{акустический излучатель (мембрана)}, преобразуя механические колебания в звуковые волны в воздухе.

Звуковая волна представляет собой колебания давления с доминирующей частотой $f_0$ (основной тон) и обертонами $3f_0, 5f_0$. Эта волна воздействует на барабанную перепонку уха. Во внутреннем ухе (улитке) происходит спектральный анализ: разные участки базилярной мембраны резонируют на разные частоты. Мозг интерпретирует стабильное возбуждение участка, соответствующего частоте $f_0$, как восприятие \textbf{чистого тона} определённой высоты.

\subsection{Качественные предсказания теории и их проверка}

Из формулы (\ref{eq:fundamental_frequency_rod}) следуют проверяемые следствия:
\begin{itemize}
    \item \textbf{Зависимость от длины ($L$)}: $f_0 \propto 1/L$. Чем короче выступающая часть мела, тем выше тон скрипа. Это легко наблюдается на опыте.
    \item \textbf{Зависимость от материала}: $f_0 \propto \sqrt{E/\rho}$.
    \begin{itemize}
        \item Более твёрдый и упругий мел (больше $E$) скрипит \textbf{выше}.
        \item Более плотный мел (больше $\rho$) скрипит \textbf{ниже}.
        \item Влажный или рыхлый мел имеет меньший эффективный модуль Юнга $E$ и большее затухание, что приводит к более низкому, глухому звуку или отсутствию чистого тона.
    \end{itemize}
    \item \textbf{Роль трения}: От коэффициента трения $\mu$ зависит эффективность режима "stick-slip". При очень низком трении (например, на гладкой или смазанной поверхности) срывы не происходят, стоячая волна не возбуждается — мел скользит бесшумно.
\end{itemize}

\section{Заключение}

В представленной работе проведено последовательное физико-математическое моделирование явления скрипа мела, раскрывающее его природу как сложного \textbf{автоколебательного процесса}.

Установлено, что \textbf{первичным источником колебаний} служит нелинейное трение «сцепление-скольжение» (stick-slip). Математический анализ с использованием модели трения с отрицательным наклоном зависимости от скорости ($dF/dv < 0$) показал, что эта система обладает динамической неустойчивостью, переводящей постоянное движение руки в периодические ударные импульсы — типичный признак \textbf{автогенератора}.

Показано, что эти импульсы возбуждают в упругом стержне мела широкий спектр упругих волн. В результате многократных отражений от концов, описываемых с позиций \textbf{акустического импеданса} ($Z = \sqrt{\rho E}$) и коэффициента отражения ($R$), в системе формируется стоячая волна.

Теоретически выведена и численно оценена формула для основной собственной частоты продольных колебаний стержня с одним закреплённым концом: $f_1 = \frac{1}{4L} \sqrt{\frac{E}{\rho}}$. Полученное значение (~6 кГц для типичных параметров) соответствует высокочастотному звуку. Объяснена структура спектра, включающего только \textbf{нечётные гармоники} ($f_n = (2n+1)f_1$).

Рассмотрены реалистичные граничные условия и введён \textbf{коэффициент стоячей волны (КСВ)} как мера «чистоты» стоячей волны. Конечное значение КСВ объясняет наличие минимальной (ненулевой) амплитуды в узлах и существование бегущей компоненты, ответственной за перенос энергии от источника к поглотителям.

Объяснён механизм эффективного акустического излучения. Показано, что доска служит \textbf{импедансным трансформатором}, компенсируя огромное рассогласование импедансов мела и воздуха за счёт большой излучающей площади, и преобразуя точечные колебания в мощную звуковую волну.

Таким образом, скрип мела — это не просто трение, а цепочка преобразований энергии: \textbf{механическая работа руки $\to$ энергия упругой деформации в режиме stick-slip $\to$ энергия стоячих волн в стержне $\to$ энергия изгибных колебаний доски $\to$ акустическая энергия в воздухе}. Данное явление является наглядным и содержательным примером синтеза теорий трения, упругости, волновой физики, акустики и теории колебаний.
